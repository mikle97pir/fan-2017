\documentclass{notes}
\usepackage[bb=boondox]{mathalfa} % чтобы делать двойные цифры

\DeclareMathOperator{\esssup}{esssup}
\DeclareMathOperator{\supp}{supp}
\DeclareMathOperator{\Int}{Int}
\newcommand{\mink}{\mathfrak{m}}


\begin{document}

	\begin{center}
		\huge{\sffamily{Функциональный анализ--1}} \\
		\vspace{0.5em}
		\large{Михаил Пирогов}
	\end{center}
	\vspace{0.5em}
	\abstract{Краткое содержание курса А. Д. Баранова, прочитанного в осеннем семестре 2017 года.}

\section{Топологические векторные пространства}

\subsection{Основные определения}

	\begin{de}
		Пусть $X$~--- векторное пространство над $\R$ или $\C$, снабжённое топологией $\tau$. Пару $(X, \, \tau)$ называют \ti{топологическим векторным пространством,} если сложение и умножение на скаляр непрерывны относительно $\tau$, и каждая точка является замкнутым множеством.
	\end{de}

	\begin{exm}
		Нормированное пространство со стандартной топологией~--- ТВП, а прямая с топологией Зарисского~--- нет, ибо в ней $U + V = \R$ для любых двух открытых множеств (ведь их дополнения конечны).
	\end{exm}

	\begin{st}
		Параллельный перенос $T_a$ и растяжение $M_{\lambda}$~--- гомеоморфизмы ТВП $X$ в себя. При $T_a$ локальная база переходит в локальную базу.
	\end{st}

	\begin{de}
		ТВП называют \ti{локально выпуклым,} если в нём есть база в нуле, состоящая из выпуклых множеств.
	\end{de}

	\begin{de}
		Множество $E \subset X$ называют \ti{ограниченным,} если для любой окрестности нуля $U$
		\[
			\ex t \col \; \all s > t \; sU \supset E.
		\] 
	\end{de}

	\begin{st}
		Легко видеть, что при растяжении ограниченное множество переходит в ограниченное множество. Позже мы докажем, что это происходит с ним при любом непрерывном линейном отображении (то есть и при параллельном переносе).
	\end{st}

	\begin{de}
		ТВП $X$ называют \ti{локально ограниченным,} если в нём существует база в нуле из ограниченных множеств.
	\end{de}

	\begin{st}
		Существования всего одной ограниченной окрестности нуля достаточно, чтобы ТВП было локально ограничено.
	\end{st}

	\begin{thm}
		ТВП $(X, \, \tau)$ метризуемо $\eqv$ есть счётная база в нуле. 
	\end{thm}

	\begin{thm}[Колмогоров]
		ТВП нормируемо $\eqv$ оно локально ограничено и локально выпукло.
	\end{thm}

\subsection{Топология, порождённая счётным семейством полунорм}

	\begin{de}
		Пусть $X$~--- векторное пространство над $\R$ или $\C$. Функцию 
		$p\col \; X \to [0, \, \infty)$ называют \ti{полунормой}, если выполняются следующие условия:
		\begin{enumerate}
			\item $p(\lambda x) = |\lambda| p(x)$,
			\item $p(x + y) \leqslant p(x) + p(y)$.
		\end{enumerate}
	\end{de}

	\begin{exm}
		На $C\big((-1, \, 1)\big)$ полунормой является
		\[
			\|f\| = \max\limits_{\left[-\tfrac{1}{2}, \, \tfrac{1}{2}\right]} |f|.
		\]
	\end{exm}

	\begin{de}
		Семейство полунорм $\{p_n\}_{n \in \N}$ на ВП $X$ называют \ti{определяющим,} если 
		\[
			\all n \; p_n(x) = 0 \so x = 0.
		\] 
	\end{de}

	\begin{de}
		Топологией, \ti{порождённой} семейством полунорм, называют самую грубую топологию, относительно которой все они непрерывны. 
	\end{de}

	\begin{st}
		Базой этой топологии являются множества вида
		\[
			V_{\varepsilon, \, i_1, \,  \ldots, i_n}(x_0) = \big\{x \in X \, \big| \, \all k \; p_{i_k}(x - x_0) < \varepsilon \big\}.
		\]
	\end{st}

	\begin{st}
		Семейство полунорм определяющее $\eqv$ топология, порождённая им, хаусдорфова.
	\end{st}

	\begin{st}
		Векторное пространство с топологией, порождённой определяющим семейством полунорм~--- локально выпуклое ТВП.
	\end{st}

	\begin{thm}
		Топология $\tau$, порождённая определяющим семейством полунорм $p_n$, задаётся метрикой
		\[
			\rho(x, \, y) = \sum \limits_{n = 1}^{\infty} \dfrac{\min(1, \, p_n(x - y))}{2^n}.
		\]
		\begin{proof}
			$\hphantom{.}$
			\begin{enumerate}

				\item Очевидно, что ряд сходится, причём $\rho(x, \, y) \geqslant 0$.
				\item Если $\rho(x, \, y) = 0$, то все слагаемые нулевые, поэтому $x = y$.
				\item $\rho(x,\, y) = \rho(y, \, x)$.
				\item Если верно неравенство
				\[
					\all a, \, b \geqslant 0 \; \min (1, \, a + b) \leqslant \min(1, \, a) + \min(1, \, b),
				\]
				то 
				\[
					\min(1, \, p_n(x - z)) \leqslant \min\big(1, \, p_n(x - y) + p_n(y - z)\big) \leqslant \min(1, \, p_n(x - y)) + \min(1, \, p_n(y - z)),
				\]
				откуда сразу следует искомый результат.
				\item Неравенство довольно просто доказать.
				\item Осталось лишь понять, что этой метрикой задаётся нужная топология. Для этого достаточно доказать, что
				\[
						\all B_{\delta}(0) \; \ex V_{\varepsilon, \, i_1, \, \ldots, \, i_n}(0) \col \, V_{\varepsilon} \subset B_{\delta}
				\]
				и, напротив,
				\[
						\all V_{\varepsilon, \, i_1, \, \ldots, \, i_n}(0) \; \ex B_{\delta}(0)  \col  \, B_{\delta} \subset \, V_{\varepsilon}.
				\]

				Докажем сначала первое включение. Найдётся такое $N$, что
				\[
					\all N' > N \; \sum\limits_{n = N'}^{\infty} \dfrac{1}{2^n} < \dfrac{\delta}{2} \so \all x \in B_{\delta} \; \sum\limits_{n = N'}^{\infty} \dfrac{\min\big(1, \, p_n(x)\big)}{2^n} < \dfrac{\delta}{2}.
				\]
				Возьмём $\varepsilon = \dfrac{\delta}{2}$ и $n = N$. Тогда
				\[
					x \in V_{\varepsilon} \so \all k 
					\leqslant N \; p_k(x) < \dfrac{\delta}{2}.
				\]
				Поэтому
				\[	
					\sum\limits_{n = 1}^{N} \dfrac{\min\big(1, \, p_n(x)\big)}{2^n} \leqslant \sum\limits_{n = 1}^{N} \dfrac{p_n(x)}{2^n} < \dfrac{\delta}{2}\sum\limits_{n = 1}^{N} \dfrac{1}{2^n} < \dfrac{\delta}{2}. 
				\]
				Таким образом получаем, что из того, что $x \in V_{\varepsilon}$, следует, что
				\[
					\sum \limits_{n = 1}^{\infty} \dfrac{\big(1, \, p_n(x)\big)}{2^n} < \delta \so x \in B_{\delta}.
				\]
				\item Докажем теперь второе включение. Не умаляя общности, положим $\varepsilon < 1$. Пусть $\max(i_1, \, \ldots, \, i_n) = N$. Возьмём
				\[
					\delta = \dfrac{\varepsilon}{2^N}.
				\]
				Если $x \in B_{\delta}$, то
				\[
					\sum\limits_{n = 1}^{\infty} \dfrac{p_n(x)}{2^n} < \dfrac{\varepsilon}{2^N} \so \sum\limits_{n = 1}^{N} \dfrac{p_n(x)}{2^n} < \dfrac{\varepsilon}{2^N} \so \all n \leqslant N \; p_n(x) < \varepsilon.
				\] 
				Отсюда получаем, что $x \in V_{\varepsilon}$.
			\end{enumerate}
		\end{proof}
	\end{thm}

	\begin{thm} \label{thm:seminorms-prop}
		Пусть $X$~--- ТВП с топологией $\tau$, порождённой определяющим семейством полунорм $p_n$. 
		\begin{enumerate}
			\item $x_k \to x_0 \eqv \all n \; p_n(x_k - x_0) \to 0$,
			\item $E \subset X$ ограничено $\eqv \all n \; p_n$ ограничены на $E$. 
		\end{enumerate}
		\begin{proof}
			$\hphantom{.}$
			\begin{enumerate}
				\item $\so\col$ Пусть $x_k \to x_0$. Это означает, что $\rho(x_k, \, x_0) \to 0$, т.е.
				\[
					\sum\limits_{n = 1}^{\infty} \dfrac{\min\big(1, \, p_n(x_n - x_0)\big)}{2^n} \to 0 \so \all n \; p_n(x_k - x_0) \to 0.
				\]

				$\bso\col$ Пусть все $p_n$ стремятся к нулю. Рассмотрим $\varepsilon > 0$. Пусть $N$ таково, что
				\[
					\sum\limits_{n = N + 1}^{\infty} \dfrac{1}{2^n} < \dfrac{\varepsilon}{2}.
				\]
				Тогда 
				\[
					\rho(x_k, \, x_0) < \sum\limits_{n = 1}^{N} \dfrac{\min\big(1, \, p_n(x_k - x_0)\big)}{2^n} + \dfrac{\varepsilon}{2}.
				\]
				Выбирая достаточно большое $k$, первую сумму можно сделать меньше $\dfrac{\varepsilon}{2}$.

				\item $\so\col$ Пусть множество $E$ ограничено. Фиксируем некоторую полунорму $p_n$ из семейства; рассмотрим окрестность $V_{\varepsilon, \, n}(0)$. Т.к. $V$ является окрестностью нуля, $E \subset kV$ для некоторого $k$. Но тогда $p_n(x) < k$ для любого $x$ из $E$.

				$\bso\col$ Пусть теперь все полунормы ограничены на $E$. Возьмём $U$~--- произвольную окрестность нуля, и $V_{\varepsilon, \, i_1, \, \ldots, \, i_n}(0) \subset U$. Найдутся $M_i$ такие, что $\all x \in E \; p_i(x) < M_i$. Отсюда следует, что $E \in nU$, если $n > M_in_i$ для всех $i$. Поэтому $E$ ограничено. Если умножить $V$ на число, превосходящее $M_{i_1}, \, \ldots\, M_{i_n}$, то получится окрестность, содержащая $E$.  
			\end{enumerate} 
		\end{proof}
	\end{thm}

	\begin{exm}
		Примеры~--- $C\big((a, \, b)\big)$, $C^{\infty}(\Omega), \; \Omega \subset \R^n$~--- открытое множество. На $C^{\infty}(\Omega)$ нужно построить последовательность компактов $K_n$ такую, что $K_n \subset \Int K_{n+1}$ и $\cup K_n = \Omega$. После этого полунорма $p_n$ определяется следующим образом:
		\[
			p_n(f) = \max\limits_{x \in K_n, \; |\alpha| \leqslant n} \big|D^{\alpha} f(x)\big|.
		\]
		Оба этих пространства полны. 

		Простой пример~--- множество последовательностей комплексных чисел $\{x_n\}$, в котором $p_n$ возвращает модуль $x_n$. Его обозначают $\C^{\infty}$.

		Гораздо больше написано в параграфе <<Примеры>> первой главы книжки Рудина.
	\end{exm}

\subsection{Функционал Минковского}
	
	\begin{de}
		Пусть $X$~--- ТВП, $A \subset X$. $A$ называют \ti{поглощающим}, если 
		\[
			\all x \in X \; \ex t > 0 \col \; x \in tA.
		\]
	\end{de}

	\begin{rem}
		Если $A$ поглощающее, то $0 \in A$.
	\end{rem}

	\begin{st}
		Любая окрестность нуля~--- поглощающее множество.
		\begin{proof}
			Это можно вывести, например, из того, что ноль~--- ограниченное множество, а точка~--- ноль после параллельного переноса. Параллельный перенос, как непрерывное линейное оторбажение, сохраняет ограниченность.

			Иначе это можно увидеть так: понятно, что $x \cdot 0 = 0$, а из непрерывности умножения следует, что
			\[
				\all U(0) \; \ex V(x), \, W_{\varepsilon}(0)\col \; V W_{\varepsilon} \subset U.
			\]
			Поэтому
			\[
				\dfrac{1}{t} \in W_{\varepsilon} \so \dfrac{x}{t} \in U \so x \in tU.
			\]
		\end{proof}
	\end{st}
h
	\begin{de}
		Пусть $A$~--- поглощающее множество. Тогда 
		\[
			\mink_A(x) = \inf \left\{t \, \big| \, \dfrac{x}{t} \in A\right\} \text{~--- функционал Минковского.}
		\]
	\end{de}

	\begin{rem}
		Если $A$ выпукло и содержит ноль, то из того, что $\tfrac{x}{t} \in A$, следует, что $\tfrac{x}{s} \in A$ для любого $s > t$.
	\end{rem}

	\begin{st}
		Пусть $A$~--- выпуклое и поглощающее, и $t > \mink_A(x)$. Тогда $\tfrac{x}{t} \in A$.
	\end{st}

	\begin{thm}\label{thm:mink-func}
		Пусть $A$~--- выпуклое поглощающее множество. Тогда:
		\begin{enumerate}
			\item $\all t > 0 \; \mink_A(tx) = t\mink_A(x)$ (положительная однородность),
			\item  $\mink_A(x + y) \leqslant \mink_A(x) + \mink_A(y)$ (полуаддитивность),
			\item Если $A$ уравновешенное, $\mink_A$~--- полунорма.
		\end{enumerate}
		\begin{proof}
			$\hphantom{.}$
			\begin{enumerate}
				\item 
				\[
					\mink_A(tx) = \inf \left\{s \, \big| \, \dfrac{tx}{s} \in A\right\} = t \inf \left\{\dfrac{s}{t} \, \big| \, \dfrac{tx}{s} \in A\right\} = t \mink_A(x).
				\]
				\item Для любого $\varepsilon > 0$ найдутся $s$ и $t$ такие, что
				\[
					s - \varepsilon < \mink_A(x) < s, \; \dfrac{x}{s} \in A \text{ и } t - \varepsilon < \mink_A(y) < t \text{ и } \dfrac{y}{t} \in A.
				\]
				Распишем частное сумм:
				\[
					\dfrac{x + y}{s + t} = \dfrac{s}{s + t} \dfrac{x}{s} + \dfrac{t}{s + t} \dfrac{y}{t}.
				\]
				По выпуклости это частное лежит в $A$, поэтому
				\[
					\mink_A(x + y) \leqslant s + t < \mink_A(x) + \mink_B(y).
				\]
				\item Не хватает возможности умножать на любую константу из $\C$. Пусть $\alpha = r\beta$, $r \geqslant 0$, $|\beta| = 1$.
				\[
					\mink_A(\alpha x) = \inf \left\{s \, \big| \, \dfrac{\alpha r x}{s} \in A\right\} = \inf \left\{s \, \big| \, \dfrac{ r x}{s} \in \underbrace{\alpha^{-1} A}_{=A}\right\} = \mink_A(rx) = r \mink_A(x) = |\alpha| \mink_A(x).
				\]
			\end{enumerate}
		\end{proof}
	\end{thm}

\subsection{Теорема о нормируемости}

	\begin{lm}
		$\hphantom{.}$
		\begin{enumerate}
			\item Любая окрестность нуля содержит уравновешенную окрестность;
			\item любая выпуклая окрестность нуля содержит выпуклую уравновешенную окрестность.
		\end{enumerate}
		\begin{proof}
			$\hphantom{.}$

			\begin{enumerate}
				\item Пусть $U$~--- окрестность нуля. По непрерывности умножения, существуют $V(0)$ и $\varepsilon > 0$ такие, что если $|\alpha| < \varepsilon$, то $\alpha V \subset U$. 
				Объединение $\alpha V$ по всем $\alpha$ и есть искомая окрестность.
				\item Пусть $U$~--- выпуклая окрестность нуля. Положим $A = \cap \alpha U$ по всем $\alpha$ на единичной окружности. Пусть $W$~--- окрестность из предыдущего пункта. Очевидно, что $W \subset A$. Отсюда следует, что внутренность $\Int A$ является окрестностью нуля, лежащей в $U$. Выпуклость и уравновешенность внутренности следуют из выпуклости и уравновешенности $A$.
			\end{enumerate}
		\end{proof}
	\end{lm}

	\begin{thm}[Колмогоров]
		Следующие условия равносильны:
		\begin{enumerate}
			\item $X$ нормируемо;
			\item $X$ локально выпукло и локально ограничено.
		\end{enumerate}
		\begin{proof}
			$\hphantom{.}$

			$1 \so 2\col$ Очевидно.

			$2 \so 1 \col$ Пусть $\{U_{\alpha}\}$~--- база в нуле из выпуклых окрестностей, $V$~--- ограниченная окрестность. Найдётся $\alpha$ такое, что $U_{\alpha} \subset V \so U_{\alpha}$ ещё и ограниченная. Выпуклая окрестность нуля $U_{\alpha}$ содержит выпуклую уравновешенную $U$; таким образом, $U$~--- выпуклое ограниченное уравновешенное поглощающее множество, окрестность нуля. Введём норму следующим образом:
			\[
				\|x\| = \mink_U(x).
			\]
			По теореме \ref{thm:mink-func} $\|\cdot\|$~--- полунорма.

			Докажем, что это норма. Возьмём $x \neq 0$. Все точки замкнуты, поэтому существует окрестность нуля, не содержащая $x$. Т.к. $U$ ограничена, найдётся $r > 0$ такое, что $rU$ лежит в этой окрестности. Значит, есть $r$ такое, что 
			\[
				x \notin rU \underset{\text{вып.}}{\so} \all s \in (0, r) \; \dfrac{x}{s} \notin U \so \mink_U(x) \geqslant r > 0 \so \|x\| \neq 0.
			\]

			Осталось доказать, что эта норма задаёт нужную топологию. Для этого достаточно получить, что
			\[
				rU = \left\{x \, \big| \, \|x\| < r\right\};
			\]
			отсюда будет следовать совпадение локальных баз. Докажем это равенство.

			Обозначим множество справа через $B_r$. Заметим, что
			\[
				x \in rU \so \dfrac{x}{r} \in U \so \|x\| \leqslant r.
			\]
			Отсюда $rU \subset \ov{B_r}$. Т.к. $rU$ открытое, $rU \subset B_r$.

			Докажем обратное включение. 
			\[
				\|x\| < r \so \ex s < r\col \; \dfrac{x}{s} \in U.
			\]
			Поскольку $U$ выпукло
			\[
				\dfrac{x}{r} \in U \so x \in rU.
			\]
			



		\end{proof}
	\end{thm}

\subsection{Примеры ненормируемых пространств}
	
	\begin{st}
		Пусть на локально ограниченном $X$ топология задана определяющим семейством полунорм $\{p_n\}$. Тогда найдётся окрестность нуля $V_{\varepsilon, \, 1, \, \ldots, \, n}(0)$ такая, что на ней все полунормы ограничены.
		\begin{proof}
			Пусть $X$ локально ограничено. Тогда найдётся ограниченная $V_{\varepsilon, \, 1, \, \ldots, \, n}(0)$. По теореме \ref{thm:seminorms-prop} это как раз и значит, что
			\[
				\all i \in \N \; \sup\limits_{V} p_i < \infty.
			\]
		\end{proof}
	\end{st}

	\begin{exm}
		Легко видеть, что для пространств $C^{\infty}(\Omega)$, $C(\Omega)$, $\C^{\infty}$ это всегда не так.
	\end{exm}

	\begin{de}
		Говорят, что ТВП $X$ обладает \ti{свойством Гейне-Бореля,} если любое замкнутое и ограниченное множество в нём компактно. 
	\end{de}

	\begin{rem}
		Из теоремы Рисса следует, что любое нормированное пространство, обладающее этим свойством, конечномерно.
	\end{rem}

	\begin{st}
		$C^{\infty}$ обладает свойством Гейне-Бореля.
		\begin{proof}	
			Поскольку $C^{\infty}$ метризуемо, в нём компактность равносильна секвенциальной компактности. Её и будем проверять.

			Пусть $x^k = \big(x_n^k\big)_{n = 1}^{\infty}$~--- элементы $\C^{\infty}$. Тогда сходимость $x^k$ к $x^0$ просто означает, что
			\[
				\all n \; x_n^k \to x_n^0.
			\]

			Пусть $E$~--- замкнутое и ограниченное подмножество $X$, $x^k \in E$. $E$ ограничено $\so$ $\all n \; p_n(x^k) = |x_n^k|$ ограничены. Поэтому можно выделить $x^{k, \, 1}$~--- подпоследовательность в $\{x^k\}$ такую, что $x^{k, \, 1}_n$ сходится. Продолжая диагональным методом, получим то, что нужно.
		\end{proof}
	\end{st}

	\begin{rem}
		$C^{\infty}(\R)$ обладает свойством Гейне-Бореля, а $C(\R)$ нет. $\mc{H}(\C)$ обладает свойством Гейне-Бореля.
	\end{rem}



\end{document}