\documentclass{notes}
\usepackage[bb=boondox]{mathalfa} % чтобы делать двойные цифры

\DeclareMathOperator{\esssup}{esssup}
\DeclareMathOperator{\supp}{supp}
\DeclareMathOperator{\Int}{Int}
\DeclareMathOperator{\Ker}{Ker}
\DeclareMathOperator{\Lin}{Lin}
\DeclareMathOperator{\Conv}{Conv}
\DeclareMathOperator{\Ext}{Ext}
\DeclareMathOperator{\re}{Re}


\newcommand{\mink}{\mathfrak{m}}
\newcommand{\ta}{$*$}
\newcommand{\ssph}{\ov{B}^*}
\newcommand{\weak}{\xrightarrow{w}}
\newcommand{\sweak}{\xrightarrow{w^{*}}}
\newcommand{\cint}[1]{\,\stackrel{\circ}{#1}}


\begin{document}

	\begin{center}
		\huge{\sffamily{Функциональный анализ--1}} \\
		\vspace{0.5em}
		\large{Михаил Пирогов}
	\end{center}
	\vspace{0.5em}
	\abstract{Конспект курса А. Д. Баранова, прочитанного в осеннем семестре 2017 года.}

\section{Топологические векторные пространства}

\subsection{Основные определения}

	\begin{de}
		Пусть $X$~--- векторное пространство над $\R$ или $\C$, снабжённое топологией $\tau$. Пару $(X, \, \tau)$ называют \ti{топологическим векторным пространством,} если сложение и умножение на скаляр непрерывны относительно $\tau$, и каждая точка является замкнутым множеством.
	\end{de}

	\begin{exm}
		Нормированное пространство со стандартной топологией~--- ТВП, а прямая с топологией Зарисского~--- нет, ибо в ней $U + V = \R$ для любых двух открытых множеств (ведь их дополнения конечны).
	\end{exm}

	\begin{st}
		Параллельный перенос $T_a$ и растяжение $M_{\lambda}$~--- гомеоморфизмы ТВП $X$ в себя. При $T_a$ локальная база переходит в локальную базу.
	\end{st}

	\begin{de}
		ТВП называют \ti{локально выпуклым,} если в нём есть база в нуле, состоящая из выпуклых множеств.
	\end{de}

	\begin{de}
		Множество $E \subset X$ называют \ti{уравновешенным,} если для любого $\alpha$ такого, что $|\alpha| \leqslant 1$ верно, что $\alpha E \subset E$.
	\end{de}

	\begin{de}
		Множество $E \subset X$ называют \ti{ограниченным,} если для любой окрестности нуля $U$
		\[
			\ex t \col \; \all s > t \; sU \supset E.
		\] 
	\end{de}

	\begin{st}
		Легко видеть, что при растяжении ограниченное множество переходит в ограниченное множество. Позже мы докажем, что это происходит с ним при любом непрерывном линейном отображении (то есть и при параллельном переносе).
	\end{st}

	\begin{de}
		ТВП $X$ называют \ti{локально ограниченным,} если в нём существует база в нуле из ограниченных множеств.
	\end{de}

	\begin{st}
		Существования всего одной ограниченной окрестности нуля достаточно, чтобы ТВП было локально ограничено.
	\end{st}

	\begin{thm}
		ТВП $(X, \, \tau)$ метризуемо $\eqv$ есть счётная база в нуле. 
	\end{thm}

	\begin{thm}[Колмогоров]
		ТВП нормируемо $\eqv$ оно локально ограничено и локально выпукло.
	\end{thm}

\subsection{Топология, порождённая счётным семейством полунорм}

	\begin{de}
		Пусть $X$~--- векторное пространство над $\R$ или $\C$. Функцию 
		$p\col \; X \to [0, \, \infty)$ называют \ti{полунормой}, если выполняются следующие условия:
		\begin{enumerate}
			\item $p(\lambda x) = |\lambda| p(x)$,
			\item $p(x + y) \leqslant p(x) + p(y)$.
		\end{enumerate}
	\end{de}

	\begin{exm}
		На $C\big((-1, \, 1)\big)$ полунормой является
		\[
			\|f\| = \max\limits_{\left[-\tfrac{1}{2}, \, \tfrac{1}{2}\right]} |f|.
		\]
	\end{exm}

	\begin{de}
		Семейство полунорм $\{p_n\}_{n \in \N}$ на ВП $X$ называют \ti{определяющим,} если 
		\[
			\all n \; p_n(x) = 0 \so x = 0.
		\] 
	\end{de}

	\begin{de}
		Топологией, \ti{порождённой} семейством полунорм, называют самую грубую топологию, относительно которой все они непрерывны. 
	\end{de}

	\begin{st}
		Базой этой топологии являются множества вида
		\[
			V_{\varepsilon, \, i_1, \,  \ldots, i_n}(x_0) = \big\{x \in X \, \big| \, \all k \; p_{i_k}(x - x_0) < \varepsilon \big\}.
		\]
	\end{st}

	\begin{st}
		Семейство полунорм определяющее $\eqv$ топология, порождённая им, хаусдорфова.
	\end{st}

	\begin{st}
		Векторное пространство с топологией, порождённой определяющим семейством полунорм~--- локально выпуклое ТВП.
	\end{st}

	\begin{thm}
		Топология $\tau$, порождённая определяющим семейством полунорм $p_n$, задаётся метрикой
		\[
			\rho(x, \, y) = \sum \limits_{n = 1}^{\infty} \dfrac{\min(1, \, p_n(x - y))}{2^n}.
		\]
		\begin{proof}
			$\hphantom{.}$
			\begin{enumerate}

				\item Очевидно, что ряд сходится, причём $\rho(x, \, y) \geqslant 0$.
				\item Если $\rho(x, \, y) = 0$, то все слагаемые нулевые, поэтому $x = y$.
				\item $\rho(x,\, y) = \rho(y, \, x)$.
				\item Если верно неравенство
				\[
					\all a, \, b \geqslant 0 \; \min (1, \, a + b) \leqslant \min(1, \, a) + \min(1, \, b),
				\]
				то 
				\[
					\min(1, \, p_n(x - z)) \leqslant \min\big(1, \, p_n(x - y) + p_n(y - z)\big) \leqslant \min(1, \, p_n(x - y)) + \min(1, \, p_n(y - z)),
				\]
				откуда сразу следует искомый результат.
				\item Неравенство довольно просто доказать.
				\item Осталось лишь понять, что этой метрикой задаётся нужная топология. Для этого достаточно доказать, что
				\[
						\all B_{\delta}(0) \; \ex V_{\varepsilon, \, i_1, \, \ldots, \, i_n}(0) \col \, V_{\varepsilon} \subset B_{\delta}
				\]
				и, напротив,
				\[
						\all V_{\varepsilon, \, i_1, \, \ldots, \, i_n}(0) \; \ex B_{\delta}(0)  \col  \, B_{\delta} \subset \, V_{\varepsilon}.
				\]

				Докажем сначала первое включение. Найдётся такое $N$, что
				\[
					\all N' > N \; \sum\limits_{n = N'}^{\infty} \dfrac{1}{2^n} < \dfrac{\delta}{2} \so \all x \in B_{\delta} \; \sum\limits_{n = N'}^{\infty} \dfrac{\min\big(1, \, p_n(x)\big)}{2^n} < \dfrac{\delta}{2}.
				\]
				Возьмём $\varepsilon = \dfrac{\delta}{2}$ и $n = N$. Тогда
				\[
					x \in V_{\varepsilon} \so \all k 
					\leqslant N \; p_k(x) < \dfrac{\delta}{2}.
				\]
				Поэтому
				\[	
					\sum\limits_{n = 1}^{N} \dfrac{\min\big(1, \, p_n(x)\big)}{2^n} \leqslant \sum\limits_{n = 1}^{N} \dfrac{p_n(x)}{2^n} < \dfrac{\delta}{2}\sum\limits_{n = 1}^{N} \dfrac{1}{2^n} < \dfrac{\delta}{2}. 
				\]
				Таким образом получаем, что из того, что $x \in V_{\varepsilon}$, следует, что
				\[
					\sum \limits_{n = 1}^{\infty} \dfrac{\big(1, \, p_n(x)\big)}{2^n} < \delta \so x \in B_{\delta}.
				\]
				\item Докажем теперь второе включение. Не умаляя общности, положим $\varepsilon < 1$. Пусть $\max(i_1, \, \ldots, \, i_n) = N$. Возьмём
				\[
					\delta = \dfrac{\varepsilon}{2^N}.
				\]
				Если $x \in B_{\delta}$, то
				\[
					\sum\limits_{n = 1}^{\infty} \dfrac{p_n(x)}{2^n} < \dfrac{\varepsilon}{2^N} \so \sum\limits_{n = 1}^{N} \dfrac{p_n(x)}{2^n} < \dfrac{\varepsilon}{2^N} \so \all n \leqslant N \; p_n(x) < \varepsilon.
				\] 
				Отсюда получаем, что $x \in V_{\varepsilon}$.
			\end{enumerate}
		\end{proof}
	\end{thm}

	\begin{thm} \label{thm:seminorms-prop}
		Пусть $X$~--- ТВП с топологией $\tau$, порождённой определяющим семейством полунорм $p_n$. 
		\begin{enumerate}
			\item $x_k \to x_0 \eqv \all n \; p_n(x_k - x_0) \to 0$,
			\item $E \subset X$ ограничено $\eqv \all n \; p_n$ ограничены на $E$. 
		\end{enumerate}
		\begin{proof}
			$\hphantom{.}$
			\begin{enumerate}
				\item $\so\col$ Пусть $x_k \to x_0$. Это означает, что $\rho(x_k, \, x_0) \to 0$, т.е.
				\[
					\sum\limits_{n = 1}^{\infty} \dfrac{\min\big(1, \, p_n(x_n - x_0)\big)}{2^n} \to 0 \so \all n \; p_n(x_k - x_0) \to 0.
				\]

				$\bso\col$ Пусть все $p_n$ стремятся к нулю. Рассмотрим $\varepsilon > 0$. Пусть $N$ таково, что
				\[
					\sum\limits_{n = N + 1}^{\infty} \dfrac{1}{2^n} < \dfrac{\varepsilon}{2}.
				\]
				Тогда 
				\[
					\rho(x_k, \, x_0) < \sum\limits_{n = 1}^{N} \dfrac{\min\big(1, \, p_n(x_k - x_0)\big)}{2^n} + \dfrac{\varepsilon}{2}.
				\]
				Выбирая достаточно большое $k$, первую сумму можно сделать меньше $\dfrac{\varepsilon}{2}$.

				\item $\so\col$ Пусть множество $E$ ограничено. Фиксируем некоторую полунорму $p_n$ из семейства; рассмотрим окрестность $V_{\varepsilon, \, n}(0)$. Т.к. $V$ является окрестностью нуля, $E \subset kV$ для некоторого $k$. Но тогда $p_n(x) < k$ для любого $x$ из $E$.

				$\bso\col$ Пусть теперь все полунормы ограничены на $E$. Возьмём $U$~--- произвольную окрестность нуля, и $V_{\varepsilon, \, i_1, \, \ldots, \, i_n}(0) \subset U$. Найдутся $M_i$ такие, что $\all x \in E \; p_i(x) < M_i$. Отсюда следует, что $E \in nU$, если $n > M_in_i$ для всех $i$. Поэтому $E$ ограничено. Если умножить $V$ на число, превосходящее $M_{i_1}, \, \ldots\, M_{i_n}$, то получится окрестность, содержащая $E$.  
			\end{enumerate} 
		\end{proof}
	\end{thm}

	\begin{exm}
		Примеры~--- $C\big((a, \, b)\big)$, $C^{\infty}(\Omega), \; \Omega \subset \R^n$~--- открытое множество. На $C^{\infty}(\Omega)$ нужно построить последовательность компактов $K_n$ такую, что $K_n \subset \Int K_{n+1}$ и $\cup K_n = \Omega$. После этого полунорма $p_n$ определяется следующим образом:
		\[
			p_n(f) = \max\limits_{x \in K_n, \; |\alpha| \leqslant n} \big|D^{\alpha} f(x)\big|.
		\]
		Оба этих пространства полны. 

		Простой пример~--- множество последовательностей комплексных чисел $\{x_n\}$, в котором $p_n$ возвращает модуль $x_n$. Его обозначают $\C^{\infty}$.

		Гораздо больше написано в параграфе <<Примеры>> первой главы книжки Рудина.
	\end{exm}

\subsection{Функционал Минковского}
	
	\begin{de}
		Пусть $X$~--- ТВП, $A \subset X$. $A$ называют \ti{поглощающим}, если 
		\[
			\all x \in X \; \ex t > 0 \col \; x \in tA.
		\]
	\end{de}

	\begin{rem}
		Если $A$ поглощающее, то $0 \in A$.
	\end{rem}

	\begin{st}
		Любая окрестность нуля~--- поглощающее множество.
		\begin{proof}
			Это можно вывести, например, из того, что ноль~--- ограниченное множество, а точка~--- ноль после параллельного переноса. Параллельный перенос, как непрерывное линейное оторбажение, сохраняет ограниченность.

			Иначе это можно увидеть так: понятно, что $x \cdot 0 = 0$, а из непрерывности умножения следует, что
			\[
				\all U(0) \; \ex V(x), \, W_{\varepsilon}(0)\col \; V W_{\varepsilon} \subset U.
			\]
			Поэтому
			\[
				\dfrac{1}{t} \in W_{\varepsilon} \so \dfrac{x}{t} \in U \so x \in tU.
			\]
		\end{proof}
	\end{st}

	\begin{de}
		Пусть $A$~--- поглощающее множество. Тогда 
		\[
			\mink_A(x) = \inf \left\{t \, \big| \, \dfrac{x}{t} \in A\right\} \text{~--- функционал Минковского.}
		\]
	\end{de}

	\begin{rem}
		Если $A$ выпукло и содержит ноль, то из того, что $\tfrac{x}{t} \in A$, следует, что $\tfrac{x}{s} \in A$ для любого $s > t$.
	\end{rem}

	\begin{st}
		Пусть $A$~--- выпуклое и поглощающее, и $t > \mink_A(x)$. Тогда $\tfrac{x}{t} \in A$.
	\end{st}

	\begin{thm}\label{thm:mink-func}
		Пусть $A$~--- выпуклое поглощающее множество. Тогда:
		\begin{enumerate}
			\item $\all t > 0 \; \mink_A(tx) = t\mink_A(x)$ (положительная однородность),
			\item  $\mink_A(x + y) \leqslant \mink_A(x) + \mink_A(y)$ (полуаддитивность),
			\item Если $A$ уравновешенное, $\mink_A$~--- полунорма.
		\end{enumerate}
		\begin{proof}
			$\hphantom{.}$
			\begin{enumerate}
				\item 
				\[
					\mink_A(tx) = \inf \left\{s \, \big| \, \dfrac{tx}{s} \in A\right\} = t \inf \left\{\dfrac{s}{t} \, \big| \, \dfrac{tx}{s} \in A\right\} = t \mink_A(x).
				\]
				\item Для любого $\varepsilon > 0$ найдутся $s$ и $t$ такие, что
				\[
					s - \varepsilon < \mink_A(x) < s, \; \dfrac{x}{s} \in A \text{ и } t - \varepsilon < \mink_A(y) < t \text{ и } \dfrac{y}{t} \in A.
				\]
				Распишем частное сумм:
				\[
					\dfrac{x + y}{s + t} = \dfrac{s}{s + t} \dfrac{x}{s} + \dfrac{t}{s + t} \dfrac{y}{t}.
				\]
				По выпуклости это частное лежит в $A$, поэтому
				\[
					\mink_A(x + y) \leqslant s + t < \mink_A(x) + \mink_B(y).
				\]
				\item Не хватает возможности умножать на любую константу из $\C$. Пусть $\alpha = r\beta$, $r \geqslant 0$, $|\beta| = 1$.
				\[
					\mink_A(\alpha x) = \inf \left\{s \, \big| \, \dfrac{\alpha r x}{s} \in A\right\} = \inf \left\{s \, \big| \, \dfrac{ r x}{s} \in \underbrace{\alpha^{-1} A}_{=A}\right\} = \mink_A(rx) = r \mink_A(x) = |\alpha| \mink_A(x).
				\]
			\end{enumerate}
		\end{proof}
	\end{thm}

\subsection{Теорема о нормируемости}

	\begin{lm}
		$\hphantom{.}$
		\begin{enumerate}
			\item Любая окрестность нуля содержит уравновешенную окрестность;
			\item любая выпуклая окрестность нуля содержит выпуклую уравновешенную окрестность.
		\end{enumerate}
		\begin{proof}
			$\hphantom{.}$

			\begin{enumerate}
				\item Пусть $U$~--- окрестность нуля. По непрерывности умножения, существуют $V(0)$ и $\varepsilon > 0$ такие, что если $|\alpha| < \varepsilon$, то $\alpha V \subset U$. 
				Объединение $\alpha V$ по всем $\alpha$ и есть искомая окрестность.
				\item Пусть $U$~--- выпуклая окрестность нуля. Положим $A = \cap \alpha U$ по всем $\alpha$ на единичной окружности. Пусть $W$~--- окрестность из предыдущего пункта. Очевидно, что $W \subset A$. Отсюда следует, что внутренность $\Int A$ является окрестностью нуля, лежащей в $U$. Выпуклость и уравновешенность внутренности следуют из выпуклости и уравновешенности $A$.
			\end{enumerate}
		\end{proof}
	\end{lm}

	\begin{thm}[Колмогоров]
		Следующие условия равносильны:
		\begin{enumerate}
			\item $X$ нормируемо;
			\item $X$ локально выпукло и локально ограничено.
		\end{enumerate}
		\begin{proof}
			$\hphantom{.}$

			$1 \so 2\col$ Очевидно.

			$2 \so 1 \col$ Пусть $\{U_{\alpha}\}$~--- база в нуле из выпуклых окрестностей, $V$~--- ограниченная окрестность. Найдётся $\alpha$ такое, что $U_{\alpha} \subset V \so U_{\alpha}$ ещё и ограниченная. Выпуклая окрестность нуля $U_{\alpha}$ содержит выпуклую уравновешенную $U$; таким образом, $U$~--- выпуклое ограниченное уравновешенное поглощающее множество, окрестность нуля. Введём норму следующим образом:
			\[
				\|x\| = \mink_U(x).
			\]
			По теореме \ref{thm:mink-func} $\|\cdot\|$~--- полунорма.

			Докажем, что это норма. Возьмём $x \neq 0$. Все точки замкнуты, поэтому существует окрестность нуля, не содержащая $x$. Т.к. $U$ ограничена, найдётся $r > 0$ такое, что $rU$ лежит в этой окрестности. Значит, есть $r$ такое, что 
			\[
				x \notin rU \underset{\text{вып.}}{\so} \all s \in (0, r) \; \dfrac{x}{s} \notin U \so \mink_U(x) \geqslant r > 0 \so \|x\| \neq 0.
			\]

			Осталось доказать, что эта норма задаёт нужную топологию. Для этого достаточно получить, что
			\[
				rU = \left\{x \, \big| \, \|x\| < r\right\};
			\]
			отсюда будет следовать совпадение локальных баз. Докажем это равенство.

			Обозначим множество справа через $B_r$. Заметим, что
			\[
				x \in rU \so \dfrac{x}{r} \in U \so \|x\| \leqslant r.
			\]
			Отсюда $rU \subset \ov{B_r}$. Т.к. $rU$ открытое, $rU \subset B_r$.

			Докажем обратное включение. 
			\[
				\|x\| < r \so \ex s < r\col \; \dfrac{x}{s} \in U.
			\]
			Поскольку $U$ выпукло
			\[
				\dfrac{x}{r} \in U \so x \in rU.
			\]
			



		\end{proof}
	\end{thm}

\subsection{Примеры ненормируемых пространств}
	
	\begin{st}
		Пусть на локально ограниченном $X$ топология задана определяющим семейством полунорм $\{p_n\}$. Тогда найдётся окрестность нуля $V_{\varepsilon, \, 1, \, \ldots, \, n}(0)$ такая, что на ней все полунормы ограничены.
		\begin{proof}
			Пусть $X$ локально ограничено. Тогда найдётся ограниченная $V_{\varepsilon, \, 1, \, \ldots, \, n}(0)$. По теореме \ref{thm:seminorms-prop} это как раз и значит, что
			\[
				\all i \in \N \; \sup\limits_{V} p_i < \infty.
			\]
		\end{proof}
	\end{st}

	\begin{exm}
		Легко видеть, что для пространств $C^{\infty}(\Omega)$, $C(\Omega)$, $\C^{\infty}$ это всегда не так.
	\end{exm}

	\begin{de}
		Говорят, что ТВП $X$ обладает \ti{свойством Гейне-Бореля,} если любое замкнутое и ограниченное множество в нём компактно. 
	\end{de}

	\begin{rem}
		Из теоремы Рисса следует, что любое нормированное пространство, обладающее этим свойством, конечномерно.
	\end{rem}

	\begin{st}
		$C^{\infty}$ обладает свойством Гейне-Бореля.
		\begin{proof}	
			Поскольку $C^{\infty}$ метризуемо, в нём компактность равносильна секвенциальной компактности. Её и будем проверять.

			Пусть $x^k = \big(x_n^k\big)_{n = 1}^{\infty}$~--- элементы $\C^{\infty}$. Тогда сходимость $x^k$ к $x^0$ просто означает, что
			\[
				\all n \; x_n^k \to x_n^0.
			\]

			Пусть $E$~--- замкнутое и ограниченное подмножество $X$, $x^k \in E$. $E$ ограничено $\so$ $\all n \; p_n(x^k) = |x_n^k|$ ограничены. Поэтому можно выделить $x^{k, \, 1}$~--- подпоследовательность в $\{x^k\}$ такую, что $x^{k, \, 1}_n$ сходится. Продолжая диагональным методом, получим то, что нужно.
		\end{proof}
	\end{st}

	\begin{rem}
		$C^{\infty}(\R)$ обладает свойством Гейне-Бореля, а $C(\R)$ нет. $\mc{H}(\C)$ обладает свойством Гейне-Бореля.
	\end{rem}

\subsection{Теорема о непрерывности линейных отображений}
	
	\begin{de}
		Линейное отображение ТВП называют \ti{ограниченным,} если оно переводит ограниченные множества в ограниченные.
	\end{de}

	\begin{lm}
		\label{lm:inv-dist}
		$\hphantom{.}$
		\begin{enumerate}
			\item Если $d$~--- инвариантная относительно сдвига метрика на пространстве $X$, то для любого $x \in X$ и $n \in \N$ 
			\[
				d(nx, \, 0) \leqslant n d(x, \, 0).
			\]
			\item Если $\{x_n\}$~--- сходящаяся к $0$ последовательность точек метризуемого ТВП, то существуют такие положительные скаляры $\gamma_n$, что $\gamma_n \to \infty$ и $\gamma_n x_n \to 0$.
		\end{enumerate}
		\begin{proof}
			$\hphantom{.}$
			\begin{enumerate}
				\item 
				\[
					d(nx, \, 0) = d\big((n-1)x, \, -x\big) \leqslant d\big((n-1)x, \, 0\big) + \underbrace{d(0, \, -x)}_{=d(x, \, 0)}.
				\]
				Продолжая эту деятельность, по индукции получаем наше утверждение.
					
				\item Мы знаем, что раз $X$ метризуемо, в нём можно ввести инвариантную метрику, совместимую с топологией. Построим такую возрастающую последовательность целых чисел $n_k$, что $d(x_n, \, 0) < k^{-2}$ при $n \geqslant n_k$ и положим $\gamma_n = 1$ при $n < n_1$ и $\gamma_n = k$ при $n_k \leqslant n < n_{k + 1}$. Посмотрим, как ведёт себя последовательность $\gamma_n x_n\col$
				\[
					d(\gamma_n x_n, \, 0) = d(k x_n, \, 0) \leqslant k d(x_n, \, 0) < k^{-1}.
				\]
				Поэтому $\gamma_n x_n \to 0$ при $n \to \infty$.
			\end{enumerate}
		\end{proof}
	\end{lm}

	\begin{lm}
		\label{lm:seq-bound}
		Следующие два свойства подмножества $E$ топологического векторного пространства эквивалентны: 
		\begin{enumerate}
			\item $E$ ограничено;
			\item если $\{x_n\}$~--- любая последовательность точек из $E$, а $\alpha_n$~--- такая последовательность скаляров, что $\alpha_n \to 0$, то $\alpha_n x_n \to 0$.
		\end{enumerate}
		\begin{proof}
			$\hphantom{.}$

			$1 \so 2 \col$ Пусть $E$ ограничено, а $U$~--- произвольная окрестность нуля. По определению ограниченности
			\[	
				\big(\ex t > 0\col \; \all s > t \; E \subset sU\big) \so \left(\ex t > 0\col \; \all s > t \; \all n \; \dfrac{x_n}{s} \in U\right).
			\]
			Поскольку $\gamma_n \to 0$, с некоторого момента будет выполнено неравенство $\gamma_n < \tfrac{1}{t} \so \gamma_n x_n \in U$. По определению предела отсюда следует, что $\gamma_n x_n \to 0$.

			$2 \so 1 \col$ Пусть теперь $E$ не ограничено. Тогда найдётся окрестность нуля $U$ и последовательность скаляров $r_n \to \infty$ такие, что $E \not \subset r_n U$. Выберем $x_n$ такими, что $x_n \notin r_n U$, а $\gamma_n$ положим равным $r_n^{-1}$. Тогда $\gamma_n x_n$ ни при каком $n$ не попадает в $U$, а значит, и к нулю не сходится.
		\end{proof}
	\end{lm}

	\begin{thm} \label{thm:contin}
		Пусть $X$ и $Y$~--- ТВП, а $L\col \; X \to Y$~--- линейное отображение. Рассмотрим следующие утверждения:
		\begin{enumerate}
			\item $L$ непрерывно;
			\item $L$ ограничено;
			\item если $x_n \to 0$, то $\{Lx_n\}$~--- ограниченное множество;
			\item $x_n \to 0 \so Lx_n \to 0$.
		\end{enumerate}

		Импликация $1 \so 2 \so 3$ выполняется всегда; импликация $3 \so 4 \so 1$ выполняется, если $X$ метризуемо.

		\begin{proof}
			$\hphantom{.}$

			$1 \so 2 \col$ Пусть $E \subset X$~--- ограниченное множество, $W$~--- окрестность нуля в $Y$. Из непрерывности $L$ следует, что $L^{-1}(W)$ открыто в $X$.

			Найдётся окрестность нуля $V$ такая, что $V \subset L^{-1}(W) \so L(V) \subset W$. $E$ ограничено, поэтому существует $t$ такое, что $\all s > t \; E \subset s V \so L(E) \subset L(sV) = s L(V) \subset s W$.

			Таким образом, для произвольной окрестности $W \subset Y$ мы нашли $t$ такое, что при $s > t$ $L(E) \subset s W$. Отсюда следует ограниченность $L(E)$.

			$2 \so 3 \col$ Для этого нужно только доказать, что сходящаяся к нулю последовательность ограничена. Сходимость $x_n$ к $0$ значит, что
			\[
				\all U(0) \; \ex N \col \; \all n > N \; x_n \in U.
			\] 

			Возьмём произвольную окрестность $U$ и в ней выберем уравновешенную окрестность $V$. Из только что написанного определения следует, что вне неё находится лишь конечное количество точек последовательности; обозначим их $\left\{x_{i_k}\right\}_{k = 1}^{K}$. Поскольку любая окрестность нуля~--- поглощающее множество, для любого $k$ найдётся $n_k$ такое, что $x_{i_k} \in n_k V$. Легко видеть, что
			\[
				\{x_i\}_{i = 1}^{\infty} \subset V \cup \bigcup\limits_{k = 1}^K n_k V.
			\]
			Т.к. $V$~--- уравновешенное множество, то и $\left( \max_{k\in 1\ldots K} n_k \right) V$ тоже. Поэтому
			\[
				\{x_i\}_{i = 1}^{\infty} \subset \left(\max_{k\in 1\ldots K} n_k \right) \cdot V \subset \left(\max_{k\in 1\ldots K} n_k \right) \cdot U.
			\]
			Отсюда и следует искомая ограниченность.

			$4 \so 1 \col$ С этого места мы предполагаем, что $X$ метризуемо. Пусть (1) неверно, то есть отображение $L$ не непрерывно. Легко видеть, что оно тогда не непрерывно в нуле (для этого надо рассмотреть определение непрерывности в точке через окрестности и вспомнить, что параллельный переноc~--- автоморфизм). Это значит, что найдётся окрестность нуля $U \subset Y$ такая, что $L^{-1}(U)$ не содержит ни одной окрестности нуля.

			Поскольку $X$ метризуемо, в нём есть счётная локальная база, причём в каждом её элементе есть точка, которая не попадает в $U$. Из этих точек можно составить сходящуюся к нулю последовательность, образ которой с $U$ вовсе не пересекается. Это противоречит (3).

			$3 \so 4\col$ Пусть $X$ метризуемо и $L$ обладает свойством (3). Пусть $x_n \to 0$. По лемме \ref{lm:inv-dist} найдётся последовательность положительных скаляров $\gamma_n$ такая, что $\gamma_n \to \infty$ и $\gamma_n x_n \to 0$. Тогда $\{L \gamma_n x_n\}$~--- ограниченное множество.

			По лемме \ref{lm:seq-bound} выходит, что
			\[
				\underbrace{\dfrac{1}{\gamma_n}}_{\to 0} \underbrace{L (\gamma_n x_n)}_{\text{огранич.}} \to 0 \so Lx_n \to 0.
			\]
		\end{proof}
	\end{thm}

\subsection{Равностепенно непрерывные семейства, теорема Банаха-Штейнгауза, следствия}

	\begin{de}
		Подмножество топологического пространства называют \ti{нигде не плотным,} если его замыкание имеет пустую внутренность. Говорят, что множество относится к \ti{первой категории} (или \ti{худое}), если его можно представить в виде счётного объединения нигде не плотных множеств. Все остальные множества относят ко \ti{второй категории} (их называют ещё \ti{тучными}).
	\end{de}

	\begin{pr}
		$\hphantom{.}$
		\begin{enumerate}
			\item Если $A \subset B$ и $B$ первой категории, то $A$ тоже первой категории.
			\item Счётное объединение множеств первой категории~--- множество первой категории.
			\item Замкнутое в $S$ множество $E\subset S$ с пустой внутренностью является множеством первой категории в $S$.
			\item Если $h$~--- гомеоморфизм пространства $S$ на себя, то множества $E$ и $h(E)$ имеют одну категорию в $S$.
		\end{enumerate}
	\end{pr}

	\begin{thm}(Бэр) \label{thm:baire}
		Пусть $S$ либо
		\begin{enumerate}
			\item полное метрическое пространство, либо
			\item локально компактное хаусдорфово пространство.
		\end{enumerate}
		Тогда пересечение любого счётного семейства всюду плотных множеств всюду плотно в $S$.
	\end{thm}

	\begin{cor} \label{cor:cat}
		Полные метрические пространства и локально компактные хаусдорфовы пространства являются множествами второй категории в себе.
	\end{cor}

	\begin{de}
		Пусть $X$ и $Y$~--- ТВП, а $\Gamma$~--- некоторое семейство отображений из $X$ в $Y$. Назовём $\Gamma$ \ti{равностепенно непрерывным,} если для любой $U(0) \subset Y$ найдётся $V(0) \subset X$ такая, что $\Gamma(V) \subset U$ (т.е. $\all \Lambda \in  \Gamma \; \Lambda(V) \subset U$).
	\end{de}

	\begin{lm} \label{lm:uni-bound}
		Пусть $X$ и $Y$~--- ТВП, а $\Gamma$~--- равностепенно непрерывное семейство линейных отображений. Тогда если $E$~--- ограниченное множество в $X$, то $\Gamma(E)$ тоже ограничено.
		\begin{proof}
			Рассмотрим произвольную $U$~--- окрестность нуля в $Y$. Поскольку $\Gamma$ равностепенно непрерывно, найдётся $V(0) \subset X$ такая, что $\Gamma(V) \subset U$. Ограниченность $E$ означает, что
			\[
				\ex t \col \; \all s > t \; E \subset sV.  
			\]
			Отсюда
			\[
				\Gamma(E) \subset s\Gamma(V) \subset sU,
			\]
			что и даёт ограниченность $\Gamma(E)$.
		\end{proof}
	\end{lm}

	\begin{thm}[Теорема Банаха-Штейнгауза, принцип равномерной ограниченности] \label{thm:ban-stein}
		Пусть $X$ и $Y$~--- ТВП, $\Gamma$~--- некоторое семейство непрерывных линейных отображений из $X$ в $Y$, а $B$~--- множество всех таких точек $x \in X$, что их орбиты $\Gamma(x)$ ограничены. Если $B$~--- множество второй категории в $X$, то $B = X$ и семейство $\Gamma$ равностепенно непрерывно.
		\begin{proof}
			Выберем в $Y$ такие уравновешенные окрестности нуля $U$ и $W$, что $\ov{U} + \ov{U} \subset V$, и положим
			\[
				E = \bigcap\limits_{\Lambda \in \Gamma} \Lambda^{-1}(\ov{U}).
			\]
			Если $x \in B$, $\Gamma(x) \subset nU$ для некоторого натурального $n$, так что $x \in nE$. Поэтому
			\[
				B \in \bigcup\limits_{n \in \N} nE.
			\]
			Хотя бы одно из множеств $nE$ является множеством второй категории в $X$, ибо $B$ таково. Поскольку умножение на $n$~--- гомеоморфизм, само $E$ тоже относится ко второй категории.
			Но $E$ замкнуто, как пересечение замкнутых множеств; поэтому в нём есть внутренняя точка $x_0$. Множество $E - x_0$ содержит некоторую окрестность нуля $V$, причём
			\[
				\Lambda(V) \subset \Lambda(E) - \Lambda(x_0) \subset \ov{U} - \ov{U} \subset W
			\]
			для любого $\Lambda \in \Gamma$. 

			Отсюда следует, что $\Gamma$ равностепенно непрерывно. По лемме \ref{lm:uni-bound} $\Gamma$ ещё и равномерно ограничено, в частности, все множества $\Gamma(x)$ ограничены. Поэтому $B = X$.
		\end{proof}
	\end{thm}

	\begin{cor} \label{cor:ban-stein}
		Пусть $\Gamma$~--- семейство непрерывных линейных отображений $F$-пространства\footnote{$F$-пространство~--- ТВП, в котором топология порождается полной инвариантной метрикой.} $X$ в ТВП $Y$, причём все множества $\Gamma(x)$ ограничены. Тогда $\Gamma$ равностепенно непрерывно.
	\end{cor}

	\begin{thm} \label{thm:lim-cont}
		Пусть $X$ и $Y$~--- ТВП, а $\{\Lambda_n\}$~--- последовательность непрерывных линейных отображений из $X$ в $Y$.
		\begin{enumerate}
			\item Пусть $C$~--- множество $x \in X$, для которых $\{\Lambda_n x\}$ является последовательностью Коши в $Y$. Если $C$~--- множество второй категории в $X$, то $C = X$.
			\item Пусть $L$~--- множество всех $x \in X$ таких, что для них существует предел
			\[
				\Lambda x = \lim\limits_{n \to \infty} \Lambda_n x.
			\]
			Если $Y$~--- $F$-пространство, а $L$~--- множество второй категории в $X$, то $L = X$ и отображение $\Lambda$ непрерывно.
		\end{enumerate}
		\begin{proof}
			$\hphantom{.}$
			\begin{enumerate}
				\item
				Так как каждая последовательность Коши ограничена (для сходящихся это доказано в теореме \ref{thm:contin}; для последовательностей Коши доказательство идейно похоже\footnote{В ТВП можно назвать последовательность $\{x_n\}$ \ti{последовательностью Коши}, если для любой окрестности нуля $U$ найдётся такое $N$, что при $n, \, m > N$ точка $x_n - x_m$ лежит в $U$. Для пространств, на которых можно ввести \tb{инвариантную} метрику (на самом деле, все метризуемые таковы), это определение совпадает с обычным. Отсюда, кстати, следует, что две эквивалентные инвариантные метрики задают одинаковые последовательности Коши и полны одновременно.}), по теореме \ref{thm:ban-stein} Банаха-Штейнгауза семейство $\Lambda_n$ равностепенно непрерывно. 

				Можно проверить, что $C$~--- подпространство $X$. Его замыкание $\ov{C}$ всюду плотно (если бы это было не так, $\ov{C}$ было бы собственным подпространством $X$, поэтому у него не было бы внутренних точек и $C$ было бы первой категории). 

				Зафиксируем $x \in X$ и $W(0) \subset Y$. Из равностепенной непрерывности $\{\Lambda_n\}$ следует, что есть симметричная окрестность $V(0) \subset X$ такая, что $\Lambda_n(V) \subset W$ для всех $n$. Раз $C$ всюду плотно, найдётся точка $x' \in C \cap (x + V)$. 

				Пусть $n$ и $m$ столь велики, что
				\[
					\Lambda_n x' - \Lambda_m x' \in W.
				\]
				Тождество
				\[
					(\Lambda_n - \Lambda_m)x = \Lambda_n(x - x') + (\Lambda_n - \Lambda_m)x' + \Lambda_m(x - x')
				\]
				показывает, что $\Lambda_n x - \Lambda_m x \in W + W + W$. Поэтому $\{\Lambda_n x\}$~--- последовательность Коши в $Y$, и $x \in C$.
				\item Из полноты $Y$ следует, что $L = C$. Пусть $V$ и $W$ обозначают то же самое, что и в пункте (1); тогда $\Lambda_n(V) \subset V$ для всех $n$. Поэтому $\Lambda(V) \subset \ov{V}$. Из этого и регулярности любого ТВП (и $Y$ в том числе) следует непрерывность $\Lambda$. 
			\end{enumerate}
		\end{proof}
	\end{thm}

	\begin{rem}
		Идея доказательства пункта (1) выражается всего в трёх утверждениях:
		\begin{enumerate}
			\item $C$~--- подпространство, и его вторая категория требует, чтобы оно было всюду плотным.
			\item То, что $\Lambda_n$ равностепенно непрерывно, позволяет из того, что для $x'$ последовательность $\Lambda_n(x)$ сходится в себе, заключить это для близкой к ней $x$. 
			\item То, что $C$ всюду плотно, позволяет взять эту самую $x'$ достаточно близко к $x$.
 		\end{enumerate}
	\end{rem}

	\begin{thm} \label{thm:lim-cont-1}
		Пусть $\{\Lambda_n\}$~--- семейство непрерывных линейных отображений из $F$-пространства $X$ в ТВП $Y$, причём в каждом $x$ cуществует предел
		\[
			\Lambda x = \lim\limits_{n \to \infty} \Lambda_n x.
		\]
		Тогда отображение $\Lambda$ непрерывно.
		\begin{proof}
			Из следствия \ref{cor:ban-stein} получется, что семейство $\Lambda_n$ равностепенно непрерывно. Дальнейшее рассуждение эквивалентно пункту (2) предыдущей теоремы. 
		\end{proof}
	\end{thm}

	\begin{thm}
		Пусть $X$ и $Y$~--- ТВП, $K \subset X$~--- компактное выпуклое подмножество, а $\Gamma$~--- такое семейство непрерывных отображений из $X$ в $Y$, что для всех $x$ $\Gamma(x)$~--- ограниченное множество. Тогда $\Gamma(K)$ ограничено.
	\end{thm}

\subsection{Теорема об открытом отображении}
	
	\begin{thm} \label{thm:open-map}
		Пусть $X$~--- $F$-пространство, $Y$~--- топологическое векторное пространство, а $\Lambda\col \; X \to Y$~--- такое непрерывное линейное отображение, что его образ является множеством второй категории в $Y$. Тогда верны следующие утверждения:
		\begin{enumerate}
			\item $\Lambda(X) = Y$;
			\item $\Lambda$~--- открытое отображение;
			\item $Y$ является $F$-пространством.
		\end{enumerate}
		\begin{proof}
			Подробно изложено в \cite[с. 58--60]{R}.
		\end{proof}
	\end{thm}

	\begin{cor}[Теорема Банаха]\label{cor:ban}
		Если $\Lambda\col \; X \to Y$~--- непрерывная линейная биекция, а $X$ и $Y$~--- $F$-пространства, то $\Lambda$~--- гомеоморфизм.
		\begin{proof}
			По теореме \ref{thm:open-map} об открытом отображении отображение $\Lambda$ открыто. Из этого сразу следует непрерывность обратного к нему.
		\end{proof}
	\end{cor}

\subsection{Теорема о замкнутом графике}

	\begin{de}
		Пусть $X, \, Y$~--- множества, $f\col\; X \to Y$~--- отображение. Тогда \ti{графиком} $f$ называют множество
		\[
			\Gamma_f = \left\{(x, \, f(x)) \in X \times Y \, \big |\, x \in X\right\}.
		\]
	\end{de}

	\begin{st}
		Пусть $X, \, Y$~--- топологические пространства, $f\col\; X \to Y$~--- непрерывное отображение, $Y$ хаусдорфово. Тогда $\Gamma_f$ замкнут в топологии произведения.
		\begin{proof}
			Рассмотрим произвольную точку $(x, \, y)$ не на графике; пусть $f(x) = y_0$. Так как $Y$ хаусдорфово, существуют непересекающиеся окрестности $U(y)$ и $V(y_0)$. Так как $f$ непрерывно, существует окрестность $W(x)$ такая, что $f(W) \subset V$. Легко видеть, что
			\[
				W \times U \cap \Gamma_f \subset W \times U \cap W \times V = \varnothing.
			\]
			Таким образом, дополнение $\Gamma_f$ открыто, поэтому оно замкнуто.
		\end{proof}
	\end{st}

	\begin{thm}
		Пусть $A\col \; X \to Y$~--- линейное отображение двух $F$-пространств. Если график $A$ замкнут, то оно непрерывно.
		\begin{proof}
			Операции векторного пространства на $X \times Y$ можно определить просто покомпонентно. Пусть $d_X$ и $d_Y$~--- полные инвариантные метрики пространств $X$ и $Y$ соответственно. Метрику на $X \times Y$ можно определить следующим образом:
			\[
				d\big((x_1, \, y_1), \, (x_2, \, y_2)\big) = d_X(x_1, \, x_2) + d_Y(y_1, \, y_2).
			\]
			Можно проверить, что она будет совместима с топологией произведения, полна и инвариантна, а всё это вместе будет $F$-пространством, но это не очень интересно.

			Поскольку отображение $A$ линейно, его график будет линейным пространством. Так как он замкнут, он тоже будет $F$-пространством.

			Определим отображения $\pi_1(x, \, \Lambda x) = x$ и $\pi_2(x, \, y) = y$ из графика в соответствующие пространства. Тогда $\pi_1$ будет непрерывной биекцией между $F$-пространствами, а по теореме Банаха \ref{cor:ban} мы знаем, что тогда и обратное к нему непрерывно. Но $A = \pi_2 \circ \pi_1^{-1}$, поэтому и оно непрерывно, как композиция непрерывных отображений.
		\end{proof}
	\end{thm}

	\begin{st}
		Пусть $A\col \; X \to Y$~--- линейное отображение двух $F$-пространств. Рассмотрим последовательность $\{x_n\}$ точек из $X$ такую, что существуют пределы
		\[
			x = \lim\limits_{n \to \infty} x_n \text{ и } y = \lim\limits_{n \to \infty} y_n.
		\]
		Если для любой такой последовательности $y = Ax$, то график $A$ замкнут.
		\begin{proof}
			Поскольку все пространства метризуемы, можно говорить о замкнутости на языке последовательностей. Если график не замкнут, то существует последовательность точек графика $(x_n, \, y_n)$, сходящаяся к точке, на нём не лежащей (полнота гарантирует, что она не сходится к <<дырке>>). Так как сходимость в $X \times Y$ с определённой нами метрикой покомпонентная, это прямое противоречие.			
		\end{proof}
	\end{st}

\subsection{Теорема Хана-Банаха}
	
	\begin{de}
		Пусть $X$~--- векторное пространство над $\R$. Отображение $p\col \; X \to \R$ называют \ti{выпуклым функционалом} на $X$, если 
		\[
			\all \lambda \in [0, \, 1] \; p\big(\lambda x_1 + (1 - \lambda) x_2\big) \leqslant \lambda p(x_1) + (1 - \lambda) p(x_2).  
		\]
	\end{de}

	\begin{de}
		Пусть $X$~--- векторное пространство над $\R$. Отображение $p\col \; X \to \R$ называют \ti{положительно однородным,} если 
		\[
			\all \alpha \geqslant 0 \; p(\alpha x) = \alpha p(x).
		\]
	\end{de}

	\begin{st}
		Пусть $p$~--- выпуклый, положительно однородный. Тогда $p(x_1 + x_2) \leqslant p(x_1) + p(x_2)$.
	\end{st}

	\begin{st}
		Пусть $p$~--- выпуклый, положительно однородный и неотрицительный. Тогда он является функционалом Минковского для множества 
		\[
			A = \{x \in x \, | \, p(x) < 1\}.
		\]
	\end{st}

	\begin{thm}[Вещественная теорема Хана-Банаха] \label{thm:ch-ban-R}
		Пусть $X$~--- векторное пространство над $\R$, $p$~--- положительно однородный выпуклый функционал на $X$, $X_0$~--- подпространство $X$.

		Рассмотрим $f_0$~--- линейный функционал на $X_0$. Если $f_0(x) \leqslant p(x)$ на $X_0$, то найдётся функционал $f\col \; X \to \R$ такой, что
		\begin{enumerate}
			\item $f|_{x_0} = f_0$,
			\item $f(x) \leqslant p(x)$ на $X$.
		\end{enumerate}
	\end{thm}


	\begin{thm}[Комплексная теорема Хана-Банаха] \label{thm:ch-ban-C}
		Пусть $X$~--- векторное пространство над $\C$, $p$~--- полунорма на $X$, $X_0$~--- подпространство $X$.

		Рассмотрим $f_0$~--- линейный функционал на $X_0$. Если $|f_0(x)| \leqslant p(x)$ на $X_0$, то найдётся функционал $f\col \; X \to \C$ такой, что
		\begin{enumerate}
			\item $f|_{x_0} = f_0$,
			\item $|f(x)| \leqslant p(x)$ на $X$.
		\end{enumerate}
	\end{thm}

\subsection{Первая теорема об отделимости}

	\begin{de}
		Пусть $X$~--- векторное пространство над $\R$, $E, \, F \subset X$~--- непустые множества. Говорят, что $E$ и $F$ \ti{отделимы,} если есть линейный функционал $f\col \; X \to \R$ такой, что
		\[
			\all x \in E \; \all y \in F \; f(x) \leqslant f(y).
		\] 
		Другими словами, $f(E) \leqslant f(F)$. Говорят, что $f$ \ti{разделяет} $E$ и $F$.
	\end{de}

	\begin{st}
		Пусть $f\col \; X \to \R$~--- линейный функционал. Тогда
		\[
			\ex a \in X\col \; \{\alpha a + x \, | \, \alpha \in \R, \; x \in \Ker f\} = X.
		\]
	\end{st}

	\begin{de}
		Подпространство $Y \neq X$ такое, что есть $a \in X\col \; \Lin(a, \, Y) = X$ называют \ti{гиперподпространством}.
	\end{de}

	\begin{st}
		$Y$~--- гиперподпространство в $X$ $\so$ есть линейный функционал $f$ такой, что $\Ker f = Y$.
	\end{st}

	\begin{de}
		\ti{Гиперплоскость}~--- множество вида $x_0 + Y$, где $Y$~--- гиперподпространство.
	\end{de}

	\begin{rem}
		Понятно, что уравнение $f(x) = \alpha$ задаёт гиперплоскость. Пусть $\sup\limits_{E} f = \alpha$. Тогда эта самая гиперплоскость разделяет множества $E$ и $F$ в обычном геометрическом смысле.
	\end{rem} 
	
	\begin{de}
		Говорят, что $E$ и $F$ \ti{строго отделимы,} если существует линейный функционал $f$ такой, что
		\[
			\sup\limits_E f < \inf\limits_F f.
		\]
	\end{de}

	\begin{thm} \label{thm:sep1}
		Пусть $X$~--- ТВП, $E$ и $F$~--- непустые выпуклые множества, $\Int E \neq \varnothing$ и $F \cap \Int E = \varnothing$. Тогда найдётся \tb{непрерывный} линейный функционал, разделяющий $E$ и $F$.
		\begin{proof}
			Введём обозначение $\cint E = \Int E$. $f(x) - f(y) \leqslant 0 \eqv f(E - F) \leqslant 0$, поэтому можно просто отделять $E - F$ от нуля.

			Не умаляя общности, предположим, что $0 \in \cint{E}$ (в противном случае можно было бы сдвинуть всё на какой-нибудь вектор). Зафиксируем $y_0 \in F$. Отделимость $E - F$ и $\{0\}$ равносильна отделмиости $E - F + y_0$ и $\{y_0\}$. Введём $K = \cint{E} - F + y_0$ и докажем сначала отделимость $K$ и $\{y_0\}$.

			Множество $K$~--- выпуклая окрестность нуля. Поэтому оно поглощающее $\so$ можно рассмотреть на нём функционал Минковского. Заметим, что 
			\[
				\cint{E} \cap F = \varnothing \so 0 \notin  \cint{E} - F,
			\]
			а значит, $y_0 \notin K$. Отсюда получаем, что $\mink_K(y_0) \geqslant 1$ (здесь мы пользуемся ещё выпуклостью $K$). 

			Рассмотрим $X_0 = \Lin(y_0)$. На нём можно задать функционал $f_0$ такой, что
			\[
				\alpha y_0 \mapsto \alpha \mink_K (y_0).
			\]
			Легко видеть, что $f_0 \leqslant \mink_K$ на $X_0$, ведь
			\[
				\begin{cases}
					\alpha \geqslant 0 \so \mink_K(\alpha y_0) = \alpha \mink_K(y_0) = f_0(\alpha y_0) \\
					\alpha < 0 \so f_0(\alpha y_0) < 0, \; \mink_A(\alpha y_0) \geqslant 0.
				\end{cases}
			\]
			По теореме Хана-Банаха $f_0$ можно продолжить на всё пространство $X$ и получить линейный функционал $f$. Понятно, что $f$ разделяет $K$ и $y_0$, ведь
			\[
				x \in K \so p(x) \leqslant 1 \so f(x) \leqslant 1
			\]
			и $f(y_0) = f_0(y_0) = p(y_0) \geqslant 1$.

			Мы доказали, что $f$ разделяет $\cint{E}$ и $F$. Почему он разделяет $E$ и $F$? Пусть $x \in E$. Нетрудно доказать\footnote{Нужно рассмотреть множество $(1 - \varepsilon)x + \varepsilon \cint{E}$ и использовать выпуклость.}, что тогда\[
				\all \varepsilon \in [0, \, 1) \; (1 - \varepsilon)x \in \cint{E}.
			\]
			Мы уже знаем, что $(1 - \varepsilon) f(x) = f\big((1 - \varepsilon)x\big) \leqslant f(y)$ при $y \in F$. Предельным переходом в неравенстве получаем искомое.

			Осталось доказать непрерывность $f$. Ноль лежит в $\cint{E}$, поэтому можно выбрать уравновешенную $V(0) \subset \cint{E}$. Зафиксируем какой-нибудь $y \in F$.
			\[
				\all x \in V \; f(x) \leqslant f(y) = a.
			\]

			Если $x$ лежит в $V$, то и $-x$ лежит в $V$, поэтому $f(-x) \leqslant a$. Отсюда следует, что $a > 0$.
			Заметим, что
			\[
				f(V) \subset (-2a, \, 2a) \so V \subset f^{-1}\big((-2a, \, 2a)\big) \so \dfrac{\varepsilon}{2a} V \subset f^{-1} \big((-\varepsilon, \, \varepsilon)\big).
			\]
			Вот и получили непрерывность.

		\end{proof}
	\end{thm}


\subsection{Вторая теорема об отделимости}

	\begin{lm} \label{lm:sum-neib}
		Каждая окрестность нуля $W$ содержит симметричную окрестность $U$ такую, что $U + U \subset W$.
		\begin{proof}
			Существование двух окрестностей $V_1$ и $V_2$ таких, что $V_1 + V_2 \subset W$ следует из непрерывности сложения. Полагая
			\[
				U = V_1 \cap V_2 \cap (-V_1) \cap (-V_2),
			\]
			получим окрестность нуля $U$, обладающую нужными свойствами.
		\end{proof}
	\end{lm}

	\begin{rem}
		Понятно, что опираясь на эту лемму, можно сделать и три, и четыре, и сколько угодно таких окрестностей.
	\end{rem}

	\begin{lm} \label{lm:top-sep}
		Пусть $K$ и $C$~--- подмножества ТВП $X$, причём $K$ компактно, $C$ замкнуто и $K \cap C = \varnothing$. Тогда найдётся окрестность нуля $V$ такая, что
		\[
			(K + V) \cap (C + V) = \varnothing.
		\]
		\begin{proof}
			Если множество $K$ пусто, то утверждение тривиально. Поэтому рассмотрим $x \in K$. По только что доказанной лемме \ref{lm:sum-neib}, найдётся окрестность $V_x(0)$ такая, что $x + V_x + V_x + V_x$ не пересекается с $C$; из симметричности $V_x$ следует, что
			\[
				(x + V_x + V_x) \cap (C + V_x) = \varnothing.
			\]

			Поскольку $K$ компактно, в нём найдётся множество точек $x_1, \, \ldots\, x_n$ такое, что
			\[
				K \subset (x_1 + V_{x_1}) \cup \ldots \cup (x_n + V_{x_n}).
			\]
			Положим $V = V_{x_1} \cap \ldots \cap V_{x_n}$. Тогда
			\[
				K + V \subset \bigcup\limits_{i = 1}^n (x_i + V_{x_i} + V) \subset K + V \subset \bigcup\limits_{i = 1}^n (x_i + V_{x_i} + V_{x_i}),
			\]
			а ни одно из множеств в последнем объединении не пересекает $C + V$.
		\end{proof}
	\end{lm}

	\begin{thm} \label{thm:sep2}
		Пусть $X$~--- локально выпуклое ТВП, $E$ и $F$~--- непустые выпуклые множества, причём $E$ компактно, а $F$ замкнуто, $E \cap F = \varnothing$. Тогда $E$ и $F$ строго отделимы.
		\begin{proof}
			Лемма \ref{lm:top-sep} позволяет отделить $E$ и $F$ непересекающимися окрестностями, а локальная выпуклость позволяет сделать их выпуклыми. После этого можно просто сослаться на первую теорему об отделимости \ref{thm:sep1}.
		\end{proof}
	\end{thm}

	\begin{de}
		Если $X$~--- комплексное ТВП, то говорят, что непустые $E$ и $F$ \ti{отделимы,} если существует линейный функционал $f$ такой, что 
		\[
			\real f(E) \leqslant \real f(F).
		\]
	\end{de}

	\begin{st}
		Для $\C$ формулировки теорем в точности такие же.
		\begin{proof}
			Нужно сослаться на доказанные теоремы, рассмотрев комплексное пространство, как вещественное. Функционал $f$ определится через вещественный, как
			\[
				f(x) = \varphi(x) - i \varphi(ix).
			\]
		\end{proof}
	\end{st}

	\begin{de}	
		Пусть $X$~--- ТВП. Тогда \ti{двойственное} к нему $X^*$~--- пространство всех \tb{непрерывных} линейных функционалов на $X$.
	\end{de}

	\begin{cor}
		Если $X$~--- локально выпуклое пространство, и $x \neq y$, то найдётся непрерывный линейный функционал такой, что $f(x) \neq f(y)$ (другими словами, $X^*$ \ti{разделяет точки пространства $X$}).
	\end{cor}

\subsection{Теорема Крейна-Мильмана}

	\begin{de}
		Пусть $X$~--- ТВП, $E \subset X$~--- выпуклый непустой компакт. Тогда говорят, что $S \subset E$~--- \ti{крайнее} для $E$, если
		\[
			x, \, y \in E; \; y \notin S; \; t \in (0, \, 1) \so tx + (1 - t)y \notin S.
		\]
	\end{de}

	\begin{de}
		Если крайнее множество состоит из одной точки, эта точка называется \ti{крайней}.
	\end{de}

	\begin{thm}[Крейна-Мильмана] \label{thm:kr-mil}
		Пусть $X$~--- ТВП, $E$~--- выпуклый непустой компакт. Пусть $\Ext{E}$~--- множество крайних точек $E$. Тогда
		\[
			E = \ov{\Conv (\Ext{E})}
		\]
	\end{thm}

\subsection{Слабые топологии}

	\begin{de}
		Пусть $X$~--- множество, $Y$~--- топологическое пространство, $\mc{F}$~--- семейство отображений из $X$ в $Y$. Обозначим через $\tau_{\mc{F}}$ топологию, состояющую из всех объединений всех конечных пересечений множеств вида $f^{-1}(U)$, где $U$ открыто в $Y$, а $f \in \mc{F}$.
	\end{de}

	\begin{rem}
		Легко видеть, что эта конструкция действительно даёт топологию.
	\end{rem}

	\begin{st}
		$\tau_{\mc{F}}$~--- самая слабая топология, относительно которой все $f \in \mc{F}$ непрерывны.
		\begin{proof}
			Рассмотрим произвольную такую топологию $\tau$. Множества вида $f^{-1}(U)$ в ней открыты по определению непрерывного отображения, а их объединения и конечные пересечения~--- по определению топологии. Поэтому $\tau_{\mc{F}} \subset \tau$.
		\end{proof}
	\end{st}

	\begin{st}
		Если пространство $Y$ хаусдорфово, и семейство $\mc{F}$ разделяет точки $X$, то $(X, \, \tau_{\mc{F}})$ тоже хаусдорфово.
		\begin{proof}
			Рассмотрим две различные точки $x_1$ и $x_2$ в $X$. Раз $\mc{F}$ их разделяет, существует $f \in \mc{F}$ такое, что $f(x_1) \neq f(x_2)$. У точек $f(x_1)$ и $f(x_2)$ есть непересекающиеся окрестности, раз $Y$ хаусдорфово, и их прообразы~--- окрестности точек $x_1$ и $x_2$~--- тоже не пересекаются.
		\end{proof}
	\end{st}

\subsection{Топология, порождённая подпространством в пространстве функционалов}

	\begin{lm} \label{lm:lin-spc}
		Пусть $X_n$~--- векторное пространство, $f_1, \, \ldots, \, f_n, \, f$~--- линейные функционалы на $X$; пусть 
		\[
			N = \bigcap\limits_{i = 1}^n \Ker f_i.
		\]
		Тогда следующие утверждения эквивалентны:
		\begin{enumerate}
			\item $\ex \alpha_1, \, \ldots, \, \alpha_n\col \; f = \alpha_1 f_1 + \ldots + \alpha_n f_n$;
			\item $\ex M \col \; \all x \in X \; |f(x)| \leqslant M \cdot \max |f_i(x)|$;
			\item $f|_{N} = 0$.
		\end{enumerate}
		\begin{proof}
			Импликация $1 \so 2 \so 3$ очевидна. Докажем $3 \so 1$. Рассмотрим
			\begin{align*}
				\Pi\col \; X &\to \C^n & \\
				x &\mapsto \big(f_1(x), \, \ldots, \, f_n(x)\big);
			\end{align*}
			пусть $Y = \Pi X$. Возьмём произвольный $y = \Pi x \in Y$; определим $F\col \; Y \to \C$ следующим образом:
			\[
				F(y) = f(x).
			\]
			Для корректности нужно проверить, что если $\Pi x = \Pi x'$, то и $f(x) = f(x')$. Это так:
			\[
				\Pi(x) = \Pi(x') \so \Pi(x - x') = 0 \so x - x' \in N \so f(x) = f(x').
			\]

			$F$~--- линейный функционал на подпространстве $\C^n$, его всегда можно продолжить на всё $\C^n$, просто отправив всё лишнее в ноль, и записать в координатах:
			\[
				F(y) = \sum\limits_{i = 1}^n \alpha_i y_i \so f(x) = F(\Pi x) = \sum\limits_{i = 1}^n \alpha_i f_i(x).
			\]
		\end{proof}
	\end{lm}


	\begin{thm}
		Пусть $X$~--- векторное пространство, $X'$~--- подпространство пространства линейных функционалов на $X$, $X'$ разделяет точки $X$. Тогда $X$ с топологией, порождённой $X'$~--- локально выпуклое ТВП, а $X^{*} = X'$.
		\begin{proof}
			Мы уже знаем, что пространство $(X, \, \tau_{X'})$ хаусдорфово. Рассмотрим множества вида
			\[
				V_{\varepsilon, \, f_1, \, \ldots, \, f_n}(x_0) = \left\{x \in X \, \bigg | \, \all i \; \big|f_i(x - x_0)\big| < \varepsilon, \; f_i \in X'\right\}.
			\]
			Нетрудно проверить, что любая точка в пересечении двух множеств такого типа содержится вместе с третьим множеством того же типа. Поэтому они образуют базу некоторой топологии $\tau$.

			Из непрерывности $f_i$ в $\tau_{X'}$ следует открытость множеств $V_{\varepsilon, \, f_1, \, \ldots, \, f_n}(x_0)$. Это значит, что $\tau \subset \tau_{X'}$. Однако 
			\[
				\all f \in X' \; f^{-1}\big(B_{\varepsilon}(x_0)\big) \in \tau,
			\]
			поэтому $f$ непрерывно относительно $\tau$. Поскольку $\tau_{X'}$ самая слабая, $\tau = \tau_{X'}$.

			Непрерывность сложения, умножения на скаляр и выпуклость доказываются довольно просто теперь, когда у нас есть удобная база. 

			Осталось лишь увидеть, что нет никаких непрерывных функционалов не из $X'$. Пусть $f$ непрерывен в $(X, \, \tau_{X'})$. Тогда найдётся окрестность вида $V_{\varepsilon, \, f_1, \, \ldots, \, f_n}(0)$ такая, что
			\[
				\all x \in V \; \big|f(x)\big| < 1.
			\]
			Возьмём $y \in N$ в обозначнениях предыдущей леммы \ref{lm:lin-spc}:
			\[
				f_i(y) = 0 \so f_i(\alpha y) = 0 \so \alpha y \in V
			\]
			для любого скаляра $\alpha$. Но
			\[
				\big|f(\alpha y)\big| < 1 \so \all \alpha \; |\alpha| \big|f(y)\big| \leqslant 1.
			\]
			Отсюда следует, что $f(y) = 0$. Пользуясь леммой, получаем искомое.
		\end{proof}
	\end{thm}

\subsection{Слабая топология и слабая сходимость}

	\begin{de}
		Пусть $X$~--- нормированное пространство. Тогда \ti{слабой топологией} на нём называют самую слабую топологию, в которой все функционалы из $X^{*}$ непрерывны. Её обозначают через $\sigma(X, \, X^{*})$.
	\end{de}

	\begin{st}
		Слабая сходимость $x_n \weak x_0$ равносильна тому, что $\all f \in X^{*} \; f(x_n) \to f(x_0)$.
		\begin{proof}
			$x_n \weak x_0$ означает, что
			\[
				\all V_{\varepsilon, \, f_1, \, \ldots, \, f_n}(x_0) \; \ex N \col \; n > N \so x_n \in V.
			\]
			Если рассмотреть окрестность типа $V_{\varepsilon, \, f}$, получится в точности то, что справа.

			Докажем теперь обратно. Для всех $f \in X^{*}$ выполняется
			\[
				f(x_n) \to f(x_0) \eqv \all V_{\varepsilon, \, f}(x_0) \; \ex N \col \; n > N \so x_n \in V.
			\]
			Рассмотрим окрестность общего вида $V_{\varepsilon, \, f_1, \, \ldots, \, f_n}(x_0)$. Если записать последнее утверждение для всех окрестностей $V_{\varepsilon, \, f_i}(x_0)$ и выбрать наибольшее из полученных $N$, оно станет верным и для окрестности общего вида.
		\end{proof}
	\end{st}

	\begin{thm}	
		$x_n \weak x_0$ равносильна тому, что одновременно выполняются два утверждения:
		\begin{enumerate}
			\item $\sup \|x_n\| < \infty$,
			\item для всех $f$ в некотором всюду плотном множестве $E \subset X^{*}$ $f(x_n) \to f(x_0)$.
		\end{enumerate}
		\begin{proof}
			Пусть $f$~--- произвольный непрерывный линейный функционал на $X$. Поскольку $E$ всюду плотно, найдётся функционал $f_0 \in E$ такой, что $\|f - f_0\| < \varepsilon$. Заметим, что
			\[
				f(x_n - x_0) = f_0(x_n - x_0) + (f - f_0)(x_n - x_0),
			\]			
			поэтому
			\[
				\big|f(x_n - x_0)\big| \leqslant \big|f_0(x_n - x_0)\big| + \big|(f - f_0)(x_n - x_0)\big| \leqslant \big|f_0(x_n - x_0)\big| + \|f - f_0 \|\big(\|x_n\| + \|x_0\|\big).
			\]
			Используя ограниченность, окончательно пишем
			\[	
				\big|f(x_n - x_0)\big|  \leqslant \big|f_0(x_n - x_0)\big| + M \|f - f_0 \|.
			\]
			Первое слагаемое стремится к нулю, а второе можно сделать сколь угодно малым, верно выбрав $f_0$. Успех!
		\end{proof}
	\end{thm}

	\begin{exm}
		Пусть $x_n \in l^p$. Тогда сходимость $x_n \weak x_0$ равносильна тому, что одновременно выполняются два утверждения:
		\begin{enumerate}
			\item $\sup \|x_n\| < \infty$,
			\item $x_n^k \to x_0^k$.
		\end{enumerate}
		\begin{proof}
			Мы знаем, как устроено пространство, двойственное к $l^p$~--- это просто $l^q$, где 
			\[
				\dfrac{1}{p} + \dfrac{1}{q} = 1.
			\]
			Линейная оболочка векторов вида 
			\[
				e_k = (0, \, \ldots, \, \underbrace{1}_{k}, \, 0, \, \ldots)
			\]
			образует всюду плотное множество в $l^q$. При этом, если рассмотреть их как функционалы, то
			\[
				e_k(x) = x^k \so e_k(x_n) \to e_k(x_0).
			\]
			При линейных комбинациях векторов это, конечно, не ломается, поэтому можно спокойно использовать только что доказанную теорему.
		\end{proof}
	\end{exm}

	\begin{exm}
		Пусть $x_n \in C(K)$, где $K \subset \R^n$~--- компакт. Тогда сходимость $x_n \weak x_0$ равносильна тому, что одновременно выполняются два утверждения:
		\begin{enumerate}
			\item $\sup \|x_n\| < \infty$,
			\item $\all t \in K \; x_n(t) \to x_0(t)$.
		\end{enumerate}
		\begin{proof}
			Мы знаем, как устроены функционалы и на $C(K)$~--- любой из них имеет вид
			\[
				\varphi(x) = \int\limits_K x \D \mu,
			\]
			где $\mu$~--- некоторая регулярная борелевская комплексная мера. По теореме Лебега об ограниченной сходимости из условий теоремы следует, что
			\[
				\int\limits_K x_n \D \mu \to \int\limits_K x_0 \D \mu,
			\]
			поскольку борелевская мера компактного множества конечна\footnote{Это требование, кажется, не всегда включают в определение борелевской меры.}.

			В обратную сторону доказательство тривиально: надо в качестве функционала взять значение в точке.
		\end{proof}
	\end{exm}

	\begin{thm}
		Пусть $H$~--- гильбертово пространство. Следующие утверждения равносильны:
		\begin{enumerate}
			\item $x_n \to x_0$;
			\item $x_n \weak x_0$ и $\|x_n\| \to \|x_0\|$.
		\end{enumerate}	
		\begin{proof}
			Доказывать надо только $2 \so 1$. Распишем норму разности:
			\[
				\|x_n - x_0\|^2 = \|x_n\|^2 + (x_n, \, x_0) - (x_0, \, x_n) + \|x_0\|^2.
			\]
			Вторые два слагаемых стремятся к квадрату нормы $x_0$ из-за слабой сходимости (они ведь непрерывные линейные функционалы по сути!) Первое стремится к квадрату нормы $x_0$.
		\end{proof}	 
	\end{thm}

\subsection{Слабая ограниченность, теорема Мазура}

	\begin{thm}
		Пусть $X$~--- нормированное пространство, $E \subset X$. Следующие условия равносильны:
		\begin{enumerate}
			\item $E$ ограничено в слабой топологии;
			\item $f(E)$ ограничено для любого непрерывного на $X$ функционала;
			\item $E$ ограничено по норме.
		\end{enumerate}
		\begin{proof}
			$\hphantom{.}$

			$3 \so 2\col$ Очевидно.

			$2 \so 3\col$ Пусть $\pi_x$~--- функционал на $X^{*}$, который переводит $f$ в $f(x)$ (он, конечно, непрерывный). Ограниченность $f(E)$ означает, что множество $\{\pi_x(f) \, | \, x \in E\}$ ограничено, а это орбита $f$! Поскольку $X^{*}$~--- $F$-пространство, можно воспользоваться теоремой Банаха-Штейнгауза \ref{cor:ban-stein} и получить, что семейство $\pi_x$ равностепенно непрерывно, а потому и равномерно ограничено, т.е. $\sup \|\pi_x\| < \infty$, а $\|\pi_x\| = \|x\|$.

			$1 \so 2\col$ Пусть $f \in X^{*}$. Рассмотрим $V_{1, \, f}(0)$. Из ограниченности $E$ следует, что
			\[
				\ex s > 0 \col \; \all t > s \; E \subset tV \so \sup |f(E)| < t.
			\]

			$2 \so 1\col$ Пусть $U$~--- окрестность нуля в слабой топологии. Не умаляя общности, $U = V_{\varepsilon, \, f_1, \, \ldots, \, f_n}$. Найдутся $M_i$ такие, что $|f_i(x)| \leqslant M_i$ для всех $x$ из $E$. Пусть $M = \max M_i$; тогда при $t > \tfrac{M}{\varepsilon}$ $E \subset tU$. 
 		\end{proof}
	\end{thm}

	\begin{thm}[Мазура] \label{thm:mas}
		Пусть $X$~--- нормированное пространство, а $E \subset X$ непусто и выпукло. Тогда замкнутость $E$ в слабой и в обычной топологии равносильны.
		\begin{proof}
			Если $E$ слабо замкнуто, то оно и по норме замкнуто, ибо топология нормы сильнее. Интересно в обратную сторону.

			Пусть множество $E$ замкнуто по норме. Предположим, что существует точка $x_0 \notin E$, которая попала в слабое замыкание $E$. По второй теореме отделимости \ref{thm:sep2} точку можно отделить от замкнутого множества $E$ функционалом $f \in X^{*}$ так, что 
			\[
				\re f(x_0) > M > \sup\limits_{x \in E} \re f(x).
			\]
			Пусть 
			\[
				U = \{x \, | \, \re f(x) > M\}.
			\]
			$f$ непрерывен в слабой топологии, поэтому $\re f$ непрерывен в ней, а значит, $U$ в ней открыто. $x_0 \in U$ и $U$ не пересекает $E$, что даёт противоречие.
		\end{proof}
	\end{thm}

\subsection{\ta-слабая топология}

	В принципе, можно было бы рассмотреть слабую топологию на пространстве линейных функционалов. Однако она оказывается довольно бесполезной, потому что второе двойственное зачастую слишком большое. Вместо этого поступают иначе.

	\begin{de}
		Пусть $X$~--- нормированное пространство. Рассмотрим отображение $\pi\col \; X^* \to X^{**}$, которое точку $x$ переводит в функционал на пространстве $X^{*}$, сопоставляющий $f \in X^{*}$ его значение $f(x)$. \ti{\ta-слабой топологией} называют самую слабую топологию, относительно которой непрерывны все функционалы из множества $\pi(X)$.
		Вместо $\pi(x)$ иногда пишут $\pi_x$.
	\end{de}

	\begin{st}[Корректность]
		\label{st:cor-sweak}
		Функционал $\pi_x$ действительно непрерывен, его норма не превосходит $\|x\|$.
		\begin{proof}
			\[
				\big|\pi_x(f)\big| = \big|f(x)\big| \leqslant \|f\| \|x\|.
			\]
			Отсюда следует, что
			\[
				\|\pi_x\| = \sup\limits_x \dfrac{\big|\pi_x(f)\big|}{\|f\|} \leqslant \|x\|.
			\]
			Это значит, что $\pi_x$ ограничен $\so$ непрерывен, причём его норма не превосходит $\|x\|$.
		\end{proof}
	\end{st}

	\begin{thm}
		Пусть $X$~--- нормированное пространство.
		\begin{enumerate}
			\item Рассмотрим $X_0 \subset X$~--- линейное подпространство, $f_0\col \; X_0 \to \C$~--- непрерывный линейный функционал. Тогда найдётся непрерывный линейный функционал $f\col \; X \to \C$ такой, что $f|_{X_0} = f_0$ и $\|f\| = \|f_0\|$.
			\item Для любой точки $x_0 \in X$ найдётся $f \in X^{*}$ такой, что $\|f\| = 1$ и $f(x_0) = \|x_0\|$.
		\end{enumerate}
		\begin{proof}
			$\hphantom{.}$
			\begin{enumerate}
				\item
				Рассмотрим $p(x) = \|f_0\| \|x\|$. Это полунорма на $X$ (а если $f_0 \neq 0)$, даже норма. Понятно, что $p(x)$ ограничивает $f_0$. Поэтому по теореме Хана-Банаха существует линейный функционал $f$ на $X$ такой, что $f|_{X_0} = f_0$ и $f(x) \leqslant \|f_0\| \|x\|$. Это сразу же даёт нам ограниченность $f$ и то, что его норма не превосходит $\|f_0\|$. При этом меньше она тоже никак быть не может.
				\item Пусть $X_0 = \Lin(x_0)$. Положим $f(\alpha x_0) = \alpha \|x_0\|$. По первому пункту теоремы всё получается.
			\end{enumerate}
 		\end{proof}
	\end{thm}

	\begin{thm}
		Отображение $\pi\col \; X \to X^{**}$~--- изометрия. 
		\begin{proof}
			Теперь мы знаем, что по любой точке $x$ можно пострить функционал $f$ такой, что $\|f\| = 1$ и $f(x) = \|x\|$. Это значит, что 
			\[
				\dfrac{\big|\pi_x(f)\big|}{\|f\|} = \|x\|.
			\]
			Поэтому верхняя оценка из утверждения \ref{st:cor-sweak} достигается и $\|\pi_x\| = \|x\|$, что и значит, что отображение $\pi$~--- изометрия.
		\end{proof}
	\end{thm}

	\begin{de}
		Если каноническое вложение $\pi$~--- изоморфизм, пространство $X$ называют \ti{рефлексивным}.
	\end{de}

	\begin{exm}
		Рефлексивны все гильбертовы пространства (потому что они изоморфны своим двойственным). Рефлексивно также $L^p$ при $1 < p < \infty$, потому что двойственное к нему $L^q$, а к нему снова $L^p$. То, что именно $\pi$  задаёт этот изоморфизм, строго говоря, надо проверять, но это довольно просто. А вот $L^1$, $L^{\infty}$ и $C(K)$ не являются рефлексивными.
	\end{exm}

	\begin{st}
		База \ta-слабой топологии состоит из множеств вида
		\[
			V_{\varepsilon, \, x_1, \, \ldots, \, x_n}(f_0) = \left\{f \in V^{*} \, \big | \, \all i \in 1\ldots n \; \big|f(x_i) - f_0(x_i)\big| < \varepsilon \right\}.
		\]
	\end{st}

	\begin{st}
		$f_n \sweak f_0 \eqv \all x \; \pi_x(f_n) \to \pi_x(f_0)$, то есть, \ta-слабая сходимость~--- по сути поточечная сходимость.
	\end{st}

	\begin{thm}
		$f_n \sweak f_0 \eqv$ выполнению двух условий:
		\begin{enumerate}
			\item $\sup \|f_n\| < \infty$.
			\item Найдётся $E$~--- всюду плотное множество в $X$ такое, что $f_n(x) \to f_0(x)$ для всех $x \in E$.
		\end{enumerate}
	\end{thm}

	Всё это делается аналогично обычной слабой топологии.

	\begin{exm}
		Если $X$ рефлексивно, то \ta-слабая и слабая тополгии на $X^{*}$ совпадают. 
	\end{exm}

\subsection{Теорема Банаха-Алаоглу}

	\begin{thm}
		Пусть $X$~--- нормированное пространство. Единичный шар $\ssph$ в пространстве $X^*$ компактен и секвенциально компактен в \ta-слабой топологии. 

		\begin{proof}[Доказательство в предположении, что $X$ сепарабельно]
			Доказывать будем в два этапа:
			\begin{enumerate}
				\item Сужение \ta-топологии на $\ssph$ метризуемо,
				\item $\ssph$ секвенциально компактен.
			\end{enumerate}
			Начнём с первого.
			\begin{enumerate}
				\item Пусть $x_n$~--- счётное всюду плотное множество в $X$. $p_n(f) = \big|f(x_n)\big|$~--- полунормы в $X^*$. Поскольку $x_n$ всюду плотно, если $p_n(f) = 0$, то и в любой точке $f$ обратится в ноль из-за его непрерывности. Поэтому семейство $p_n$ определяющее. Значит, топология $\tau$, которую порождает это семейство, метризуема.

				Докажем, что
				\[
					\sigma^*|_{\ssph} = \tau|_{\ssph}.
				\]
				База топологии $\tau$ состоит из множеств вида
				\[
					V_{\varepsilon, \, i_1, \, \ldots, \, i_n}(f_0) = \left\{f \in X^* \, \bigg | \, \all k \in 1\ldots n \; \big|f(x_{i_k}) - f_0(x_{i_k})\big| < \varepsilon \right\},
				\]
				а база топологии $\sigma^*$~--- из множеств вида
				\[
					V_{\varepsilon, \, y_1, \, \ldots, \, y_n}(f_0) = \left\{f \in X^* \, \bigg | \, \all k \in 1\ldots n \; \big|f(y_1) - f_0(y_n)\big| < \varepsilon \right\}, \; x_k \in X.
				\]
				Отсюда сразу очевидно, что $\tau \subset \sigma^*$.
				Базы топологий, суженных на $\ssph$, получаются из этих просто пересечением с $\ssph$. Хочется для них получить обратное включение, а для этого хочется доказать, что
				\[
					\all y_1, \, \ldots, \, y_n \in X \; \all \varepsilon > 0 \; \ex i_1, \, \ldots, \, i_n \col \; U_{\delta, \, i_1, \, \ldots, \, i_n}(f_0) \cap \ssph \subset V_{\varepsilon, \, x_1, \, \ldots, \, x_n}(f_0).
				\]

				Поскольку множество $\{x_n\}$ плотное, для каждого $y_k$ найдётся $x_{i_k}$ такой, что
				\[
					\|y_k - x_{i_k}\| < \dfrac{\varepsilon}{3\big(1 + \|f_0\|\big)}.
				\]
				Положим $\delta = \tfrac{\varepsilon}{3}$ и рассмотрим $f \in U_{\delta} \cap \ssph$. Проверим, лежит ли $f$ в $V_{\varepsilon}$:
				\begin{align*}
					\big|f(y_k) - f_0(y_k)\big| &= \big|f(y_k) - f(x_{i_k}) + f(x_{i_k}) - f_0(x_{i_k}) + f_0(x_{i_k}) - f_0(y_k)\big| \leqslant \\ 
					&\leqslant \big|f(y_k) - f(x_{i_k})\big| + \big|f(x_{i_k}) - f_0(x_{i_k})\big| + \big|f_0(x_{i_k}) - f_0(y_k)\big| \leqslant \\
					&\leqslant \underbrace{\|f\| \, \|y_k - x_{i_k}\|}_{<\dfrac{\varepsilon}{3\big(1 + \|f_0\|\big)}} + \underbrace{\big|f(x_{i_k}) - f_0(x_{i_k})\big|}_{\leqslant\dfrac{\varepsilon}{3}} + \|f_0\| \, \underbrace{\|x_{i_k} - y_k\|}_{\mathclap{<\dfrac{\varepsilon}{3\big(1 + \|f_0\|\big)}}} < \varepsilon.
				\end{align*}
				Успех!
				\item Пусть теперь $\{f_n\} \in \ssph$, $\{x_n\}$~--- всюду плотное множество в $X$. Заметим, что
				\[
					\big|f_n(x_1)\big| \leqslant \|f_n\| \, \|x_1\| \leqslant \|x_1\|,
				\]
				поэтому $\big\{f_n(x_1)\big\}$ ограничена в $\C$. Это значит, что можно выбрать подпоследовательность $f_n^{(1)}$ такую, что $f_n^{(1)}(x_1)$ сходится. Продолжая эту деятельность и используя диагональный метод, получаем последовательность $f_n^{(n)}$, сходящуюся во всех точках $x_n$. Но раз $\{x_n\}$ всюду плотно, оно и в каждой точке $X$ будет.
			\end{enumerate}
		\end{proof}
	\end{thm}

\subsection{Банаховы алгебры}

	\begin{de}
		Пусть $A$~--- алгебра над $\C$, т.е. линейное пространство с дистрибутивным ассоциотивным умножением, коммутирующим с умножением на константу. $A$ называют \ti{банаховой,} если 
		\begin{enumerate}
			\item На $A$ есть норма, относительно которой $A$~--- банахово пространство;
			\item $\|ab\| \leqslant \|a\|\,\|b\|$;
			\item в алгебре есть единица $e$.
		\end{enumerate}
	\end{de}

	\begin{pr} \label{pr:mul-cont}
		Умножение непрерывно относительно нормы, то есть если $x_n \to x$ и $y_n \to y$, то $x_n y_n \to xy$.
		\begin{proof}
			\begin{align*}
				\|xy - x_n y_n\| &= \|xy - xy_n + x y_n - x_n y_n\| = \big\|x(y - y_n)  + y_n \|x - x_n\| \big\| \leqslant \\
				&\leqslant \|x\| \, \|y - y_n\| + \|y_n\| \, \|x - x_n\|.
			\end{align*}
			Правая часть стремится к нулю.
		\end{proof}
	\end{pr}

	\begin{pr}
		\[
			\|x^n\| \leqslant \|x\|^n.
		\]
	\end{pr}

	\begin{exm}
		$\hphantom{.}$
		\begin{enumerate}
			\item $C(K)$,
			\item $C^1\big([a, \, b]\big)$, $\|f\| = \max |f| + \max |f'|$,
			\item $L^{\infty}(X, \, \mu)$,
			\item непрерывные операторы на банаховом пространстве,
			\item алгебра матриц,
			\item $l^1(\Z)$ с умножением
			\[
				z_n = \sum x_k y_{n - k},
			\]
			\item $L^1(X)$ со свёрткой,
			\item диск-алгебра $A(D)$~--- алгебра аналитических функций на единичном круге в $\C$,
			\item алгебра $H^{\infty}$ аналитических и ограниченных функций на единичном круге.
		\end{enumerate}
	\end{exm}

	\begin{rem}
		В алгебре может не быть единицы, как, например, в $L^1$ ($\delta$-функция). Её можно добавить с помощью общей конструкции: если есть алгебра $A$, рассмотреть алгебру $\tilde{A}$ из пар $(x, \, \alpha)$, где $x \in A$ и $\alpha \in \C$. Норму надо определить, как
		\[
			\big\|(x, \, \alpha)\big\| = \|x\| + |\alpha|,
		\]
		а умножение, как
		\[
			(x, \, \alpha) \cdot (y, \, \beta) = (xy + \alpha x + \beta y, \, \alpha \beta).
		\]
		Единица будет $(0, \, 1)$.
	\end{rem}

\subsection{Обратимые элементы}

	\begin{de}
		Пусть $A$~--- банахова алгебра. Элемент $a \subset A$ называют \ti{обратимым,} если есть $a^{-1} \in A$ такой, что $a a^{-1} = a^{-1} a = e$.
	\end{de}

	\begin{st}
		$a^{-1}$ единственен.
	\end{st}

	\begin{thm}
		Пусть $A$~--- банахова алгебра, $x \in A$, $\|x\| < 1$. Тогда элемент $e - x$ обратим, причём
		\[
			(e - x)^{-1} = \sum\limits_{i = 1}^{\infty} x^n.
		\]
		\begin{proof}	
			Докажем сначала, что ряд из формулировки вообще сходится. Поскольку пространство банахово, она будет следовать из сходимости ряда
			\[
				\sum\limits_{i = 1}^{\infty} \|x\|^n.
			\]
			Но норма $x$ меньше единицы, поэтому он сходится.

			Теперь надо понять, почему он обратный. Рассмотрим произведение
			\[
				(e - x) S_n = (e - x)\sum\limits_{i = 1}^N x^i = e - x^{N+1}.
			\]
			Правая часть стремится к $e$, а левая часть стремится к $e - x$ по свойству \ref{pr:mul-cont}. С другой стороны будет то же самое, ибо многочлены коммутируют. Успех!
		\end{proof}
	\end{thm}

	\begin{thm}
		Пусть $A$~--- банахова алгебра, $x \in A$ обратим, а $\|h\| < \|x^{-1}\|^{-1}$. Тогда элемент $x - h$ обратим, причём
		\[
			\big\|(x - h)^{-1}\big\|  \leqslant \dfrac{\|x\|^{-1}}{1 - \|h\| \|x^{-1}\|}. 
		\]
		и
		\[
			\big\|(x - h)^{-1} - x^{-1}\big\| \leqslant \dfrac{\|h\| \|x^{-1}\|^2}{1 - \|h\| \|x^{-1}\|}.
		\]
		\begin{proof}
			Более-менее простые выкладки.
		\end{proof}
	\end{thm}
\addcontentsline{toc}{chapter}{Литература}
\begin{thebibliography}{}

\bibitem{R} У.$\,$Рудин <<Функциональный анализ>>, Лань, 2005

\end{thebibliography}{}


\end{document}