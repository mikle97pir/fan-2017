\documentclass{notes}
\usepackage[bb=boondox]{mathalfa} % чтобы делать двойные цифры

\DeclareMathOperator{\esssup}{esssup}
\DeclareMathOperator{\supp}{supp}


\begin{document}

	\begin{center}
		\huge{\sffamily{Функциональный анализ-1}} \\
		\vspace{0.5em}
		\large{Михаил Пирогов}
	\end{center}
	\vspace{0.5em}
	\abstract{Краткое содержание курса А. Д. Баранова, прочитанного в осеннем семестре 2017 года.}

	\section{Топологические веекторные пространства}

	\subsection{Основные определения}

	\begin{de}
		Пусть $X$~--- линейное пространство над $\R$ или $\C$, снабжённое топологией $\tau$. Пару $(X, \, \tau)$ называют \ti{топологическим векторным пространством,} если сложение и умножение на скаляр непрерывны относительно $\tau$, и каждая точка является замкнутым множеством
	\end{de}

\end{document}