\documentclass{notes}
\usepackage[bb=boondox]{mathalfa} % чтобы делать двойные цифры

\DeclareMathOperator{\esssup}{esssup}
\DeclareMathOperator{\supp}{supp}


\begin{document}

	\begin{center}
		\huge{\sffamily{Функциональный анализ--1}} \\
		\vspace{0.5em}
		\large{Михаил Пирогов}
	\end{center}
	\vspace{0.5em}
	\abstract{Краткое содержание курса А. Д. Баранова, прочитанного в осеннем семестре 2017 года.}

	\section{Топологические векторные пространства}

	\subsection{Основные определения}

	\begin{de}
		Пусть $X$~--- линейное пространство над $\R$ или $\C$, снабжённое топологией $\tau$. Пару $(X, \, \tau)$ называют \ti{топологическим векторным пространством,} если сложение и умножение на скаляр непрерывны относительно $\tau$, и каждая точка является замкнутым множеством.
	\end{de}

	\begin{exm}
		Нормированное пространство со стандартной топологией~--- ТВП, а прямая с топологией Зарисского~--- нет, ибо в ней $U + V = \R$ для любых двух открытых множеств (ведь их дополнения конечны).
	\end{exm}

	\begin{st}
		Параллельный перенос $T_a$ и растяжение $M_{\lambda}$~--- гомеоморфизмы ТВП $X$ в себя. При $T_a$ локальная база переходит в локальную базу.
	\end{st}

	\begin{de}
		ТВП называют \ti{локально выпуклым,} если в нём есть база в нуле, состоящая из выпуклых множеств.
	\end{de}

	\begin{de}
		Множество $E \subset X$ называют \ti{ограниченным,} если для любой окрестности нуля $U$
		\[
			\ex t \col \; \all s > t \; sU \supset E.
		\] 
	\end{de}

	\begin{st}
		Легко видеть, что при растяжении ограниченное множество переходит в ограниченное множество. Позже мы докажем, что это происходит с ним при любом непрерывном линейном отображении (то есть и при параллельном переносе).
	\end{st}

	\begin{de}
		ТВП $X$ называют \ti{локально ограниченным,} если в нём существует база в нуле из ограниченных множеств.
	\end{de}

	\begin{st}
		Существования всего одной ограниченной окрестности нуля достаточно, чтобы ТВП было локально ограничено.
	\end{st}

	\begin{thm}
		ТВП $(X, \, \tau)$ метризуемо $\eqv$ есть счётная база в нуле. 
	\end{thm}

	\begin{thm}[Колмогоров]
		ТВП нормируемо $\eqv$ оно локально ограничено и локально выпукло.
	\end{thm}

\end{document}